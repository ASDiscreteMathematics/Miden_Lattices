%!TeX root=miden_lattices.tex
\section{Signatures}
\label{section:signatures}
 
\subsection{Dilithium}

Dilithium is a module-LWE based signature scheme based on Fiat-Shamir with aborts.  The base ring is given 
by $R_{q,n} = \Z_q[x]/(x^n+1)$ with $q = 2^{23} - 2^{13} - 1 = 8380417$.

The public key consists of the high bits (13 lower order bits are dropped) of MLWE-samples
\[ \bt :=  \bA \bs_1 + \bs_2 \]
where $\bA$ is a $k \times \ell$  matrix over $R_{q,n}$ derived from a seed $\rho$ 
and $\bs_1, \bs_2$ vectors of length $\ell$ and $k$ make up the secret key and 
contain elements in $R_{q,n}$ with small coefficients of max size $\eta$.
A public key then contains the following data $pk = (\rho, \bt_1)$ where $\bt_1$
is obtained from $\bt$ by dropping the lowest $13$ bits.
A summary of the relevant (for verification) parameters is given in Table~\ref{tab:dilithium}.

\begin{table}\caption{Dilithium parameters used in verification}\label{tab:dilithium}
\begin{center}
\begin{tabular}{c|c|c|c}
\hline
Security level & 2 & 3 & 5 \\
\hline
$(k,\ell)$ & (4,4) & (6,5) & (8,7) \\
$\eta$ & 2 & 4 & 2 \\
$\beta$ & $78$ & $196$ & $120$ \\
$\gamma_1$ & $2^{17}$ & $2^{19}$ & $2^{19}$ \\ 
$\gamma_2$ & $(q-1)/88$ & $(q-1)/32$ & $(q-1)/32$ \\
$\omega$ & $80$ & $55$ & $75$ \\
\hline
\end{tabular}
\end{center}
\end{table}

A signature $\sigma = (\tilde{c}, \bz, \bh)$ consists of 3 components:
\bit
\item $\tilde{c}$: a challenge of $256$ bits obtained as the hash of the public key, message and $\bA \by$ where $\by$
is a vector with elements in $R_{q,n}$ with coefficients smaller than $\gamma_1$. 
\item $\bz$: a vector of length $\ell$ of polynomials with small coefficients, in particular, smaller than $\gamma_1 - \beta$.
\item $\bh$: a vector of length $k$ consisting of $k$ hint polynomials (essentially the overflows of a particular sum) whose
coefficients are ${0,1}$ and the max number of $1$'s in $\bh$ is $\omega$.
\eit

The signature verification then proceeds as follows, where $H$ is SHAKE-256 and the specification
of the other subroutines can be found in the original specification~\cite{dilithium}.
\begin{algorithm}[!ht] \label{algo:dilithium_verify}
\begin{scriptsize}
\caption{\emph{Dilithium verification}}
\KwIn{public key $pk = (\rho, \bt_1)$, message $M$, signature $\sigma = (\tilde{c}, \bz, \bh)$}
\KwOut{boolean indicating if signature is valid}
\Begin{
\everypar={\nl}
$\bA \in R_{q,n}^{k \times \ell}$ := ExpandA($\rho$)
 
$\mu \in \{0,1\}^{512} := H(H(\rho || \bt_1) || M)$

$c := SampleInBall(\tilde{c})$

$\bw_1' := UseHint_q(h, \bA \bz - 2^d \cdot c \bt_1, 2 \gamma_2)$

return $||\bz||_{\infty} < \gamma_1 - \beta$ and $\tilde{c} = H(\mu || \bw_1')$ and $\# 1's$ in $\bh$ is $\leq \omega$
}
\end{scriptsize}
\end{algorithm} 

The matrix $\bA \in R_{q,n}^{k \times \ell}$ is generated from $\rho$ but directly into the NTT domain.
In particular, every element $a_{i,j} \in R_q$ is represented in a very specific format, which is the following:
\bit
\item Let $r = 1753$ which is a $512$-th root of unity modulo $q$
\item A polynomial $a(x) \in R_q$ in NTT representation then is given by the array:
\[ [a(r_0), a(-r_0), a(r_1), a(-r_1), \ldots, a(r_{127}), a(-r_{127})] \]
\eit
where by definition $r_i = r^{brv(128 + i)}$ where $brv(k)$ denotes the 8-bit bitreversal of the number $k$.
Note that this is the same result as the bit-reversed version of the normal array
\[  [a(r), a(r^3), \ldots, a(r^{N-1})] \, .  \]


This is done to speed up step 4 in the verification procedure above, which can be computed as 
\[ NTT^{-1} (  \bA \cdot NTT(\bz)  - NTT(c) \cdot NTT(\bt_1 \cdot 2^d))  \]
which requires:
\bit
\item $\ell + k + 1$ NTTs
\item $k$ INTTs
\item $k (\ell + 1)$ pointwise multiplications of arrays of 256 elements in $\F_q$
\eit

\subsection{Falcon}

Falcon~\cite{falcon} is a lattice-based signature scheme using NTRU lattices and a hash-and-sign approach following GPV~\cite{}.
Falcon works in a cyclotomic ring $R_{q,n} =  \Z_q[x]/(x^n+1)$ with $n = 2^k$, and an NTRU public key consists 
of a polynomial $h \in R_{q,n}$ which is computed as $g \cdot f^{-1}$ in $R_{q,n}$ where $f,g \in R_{q,n}$ that have small coefficients.
Recovering $f, g$ from $h$ corresponds to the NTRU-problem.  

Falcon specifies two parameter sets $n = 512$ and $n = 1024$ corresponding to NIST security levels I and V.  The modulus 
$q = 12289 = 12 \cdot 1024 + 1$ is prime and fixed for both sets.  The coefficients of the polynomials $f,g$ are sampled
from a discrete Gaussian with standardard deviation $\sigma_{f,g} = 1.17 \sqrt{q/2n}$.

The signature of a message $m$ consists of a salt $r$ and a pair of small polynomials $(s_1, s_2)$ such that 
$s_1 + s_2 h = H(r || m)$.  Furthermore, $s_1$ can be derived fully from $m , r$ and $s_2$ so the signature
is simply given by $(r, s_2)$.  The signature $(s_1,s_2)$ must satisfy $||(s_1,s_2)||^2 \leq \lfloor \beta^2 \rfloor$ where
$\beta = 1.1 \sigma \sqrt{2n}$.  In particular, $\beta^2 = 34034726$ for Level-I parameters and $\beta^2 = 70265242$
for Level-V parameters.  To verify a Falcon signature, one proceeds as described in Algorithm~\ref{algo:falcon_verify}

\begin{algorithm}[!ht] \label{algo:falcon_verify}
\begin{scriptsize}
\caption{\emph{Falcon verification}}
\KwIn{Message $m$, signature $sig = (r,s)$, publick key $pk = h$ and bound $\lfloor \beta^2 \rfloor$}
\KwOut{boolean indicating if signature is valid}
\Begin{
\everypar={\nl}
$c := HashToPoint(r||m,q,n)$ 

$s_2 = Decompress(s, 8 \cdot sbytelen - 328)$

if $(s_2 = \perp)$ then return false

$s_1 = c - s_2 h \bmod q$

if $||(s_1,s_2)||^2 \leq  \lfloor \beta^2 \rfloor$ then return true else return false
}
\end{scriptsize}
\end{algorithm} 
