%!TeX root=miden_lattices.tex
\section{Fast arithmetic in cyclotomic rings}
\label{section:arithmetic}

\subsection{Non-native parameter sets}

The lattice based schemes Kyber, Saber, Falcon and Dilithium, which are all finalists in the NIST pqcrypto standardization
effort (and many other schemes), all rely on arithmetic in cyclotomic rings of the form
\[  R_{q,n} = \Z_q[x]/(x^n + 1)    \]
for some modulus $q$ (not necessarily prime) and $n = 2^k$, and more in particular $n = 256, 512, 1024$.
The moduli used by the different schemes are as follows:
\bit
\item Kyber: $q = 3329$
\item Saber: $q = 2^{13}$
\item Falcon: $q = 12289$
\item Dilithium: $q = 2^{23} - 2^{13} + 1$
\eit

When the modulus $q$ is chosen such that $2n | \varphi(q)$ (in many cases $q$ is a prime so then $2n | q-1$), 
it is well known that arithmetic in $R_q$ can be computed efficiently using the number theoretic transform.
This can be done natively for Falcon and Dilithium, almost natively for Kyber and with a work-around for Saber.

In Miden, the native modulus is $p = 2^{64} - 2^{32} + 1$, so $\Z_q$ in this case contains a root of unity 
of order $2^{32}$ and in particular of order $2^k$ for any $k = 1, \ldots, 32$, and we therefore get native
arithmetic in $R_{p,n}$ for all such $n = 2^k$.

\subsection{Bound on the size of $q$}

The inequality $p \gg q$ suggests that we might simulate ring multiplication in $R_{q,n}$ using NTTs modulo $p$ without running into problems arising from the reduction modulo $p$. This is indeed the case.

Since the $q$ used in the above schemes is different from the native $p$, we first need to give a bound for
the maximum modulus size $q$ for each ring $R_{q,n}$ such that we can recover the product exactly (using 
an extra modular reduction) from the product in $R_{p,n}$.

So assume we are given two elements  $a(x), b(x) \in R_{q,n}$ written as $a(x) = \sum_{i = 0}^{n-1} a_i x^i$
and $b(x) = \sum_{i = 0}^{n-1} b_i x^i$, then the product $c(x) = \sum_{i = 0}^{n-1} c_i x^i$ satisfies
\[  c_i = \sum_{j = 0}^{i} a_{i -j} b_j  - \sum_{j = i+1}^{n-1} a_{n-j+i} b_{j}  \, . \]
In particular, we have a sum of $n$ products of elements
in $\Z_q$ (with $\pm$), so we see that the maximum bound on $c_i$ is given 
\[ |c_i| < n q^2 \, , \]
assuming that all coefficients were represented in $[0,q)$.  If a symmetric interval $[q/2, q/2)$ was used to 
represent elements in $R_{q, n}$ the bound is clearly sufficient as well.  
Since we need to be able to recover this as an integer (it can be negative) to be able to reduce modulo $q$ afterwards,
it suffices that $p \geq 2 n q^2$.  For popular choices of $n$ above we therefore obtain the following upper bounds:
\begin{center}
\begin{tabular}{|c|c|}
\hline
$n$ & $\max \log_2(q)$ \\
\hline
$256$ &  27.5 \\
$512$ &   27 \\
$1024$ &   26.5 \\
\hline
\end{tabular}
\end{center}

\subsection{Lazy reduction} In the schemes that use a module structure such as Kyber, Saber and Dilithium, 
one often has to compute 
a matrix vector product $\bA \cdot \bv$ where the matrix and vector contain elements of $R_{q,n}$.
Assuming that the matrix has $\ell$ columns, we therefore could add $\ell$ such products together before
doing the final reduction modulo $q$.  The addition of $\ell$ such products simply introduces an extra 
factor of $\ell$ in the above bound.  The largest number of columns appearing in all of the above schemes
is $7$ in the case of Dilithium level 5 parameter set.  It is easy to verify that for this case we have
$p > 7 \cdot 2 \cdot 256 \cdot q^2$, so we can indeed postpone the final reduction after the addition 
of the $\ell$ products.

\subsection{Multiplication using NTT}
\label{section:multiplication-using-ntt}

Let $N$ be a power of $2$ and assume that $q-1 \equiv 0 \bmod 2N$, and let $\psi$ be a primitive $2N$-th root 
of unity and $\omega = \psi^2$ a primitive $N$-th root of unity.  The $N$-point NTT of a sequence 
$[a[0], \ldots, a[N-1]]$ is denoted as $\tilde{a} = NTT(a)$ and defined by $\tilde{a}[i] = \sum_{j = 0}^{N-1} a[j] \omega^{i j}$
for $i = 0, \ldots, n-1$.  The inverse transformation $b = INTT(\tilde{a})$ is given by 
$b[i] = \frac{1}{N} \sum_{j = 0}^{N-1} a[j] \omega^{-i j}$, which can be computed by replacing $\omega$ 
by $\omega^{-1}$ and scaling by $N$.

Since $\omega$ is an $N$-th root of unity, note that this also corresponds to the evaluation 
of the polynomial $a(x) \in \Z_q[x]/(x^N - 1)$ in $\omega$, and as such we can multiply two 
polynomials in $a(x), b(x) \in \Z_q[x]/(x^N - 1)$ by computing
\[  c(x) = INTT(NTT(a) \cdot NTT(b))  \]
where $\cdot$ denotes pointwise multiplication.

The standard approach to NTT is given in Algorithm~\ref{algo:ntt1} where BitReverse
computes an array $A$ such that $A[k] = a[BitReverse(k)]$ obtained by reversing
the binary expansion of $k$ using $b = \log_2(N)$ bits to write $k$.
In particular, if $k = k_0 \ldots k_{b-1}$ then $BitReverse(k) = k_{b-1} \ldots k_0$. 
\begin{algorithm}[!t] \label{algo:ntt1}
\begin{scriptsize}
\caption{\emph{Iterative NTT}}
\KwIn{Polynomial $a(x) \in \mathbb{Z}_q[\mathbf{x}]$ of
degree $N-1$ and $N$-th primitive root $\omega_N \in \mathbb{Z}_q$ of unity}
\KwOut{Polynomial $A(x) \in \mathbb{Z}_q[\mathbf{x}]$ = NTT($a$)}
\Begin{
\everypar={\nl}
$A \leftarrow BitReverse(a)$;

\For{$m=2$ to $N$ by $m=2m$}
{
$\omega_m \leftarrow \omega_N^{N/m}$ \;

$\omega \leftarrow 1$ \;

\For{$j=0$ to $m/2-1$}
{
\For{$k=0$ to $N-1$ by $m$}
{

$t \leftarrow \omega \cdot A[k+j+m/2]$ \;
$u \leftarrow A[k+j]$ \;
$A[k+j] \leftarrow u+t \bmod q$ \;
$A[k+j+m/2] \leftarrow u-t \bmod q$ \;
}
$\omega \leftarrow \omega \cdot \omega_m$ \;
}
}
}
\end{scriptsize}
\end{algorithm} 


The above however is not directly applicable since we need arithmetic in the ring $\Z_q[x]/(x^N+1)$.
Note that the roots of $x^N+1$ are given by $\psi^{2k + 1}$ for $k = 0, \ldots, n-1$, so we would 
need to compute the evalutions of $a(x)$ in those powers.  Since $\psi^{2k + 1} = \omega^k \psi$, 
so we could use the standard NTT above on the scaled polynomial $a(\psi x)$, and we would obtain
\[  c(\psi x) =    INTT(NTT(a(\psi x)) \cdot NTT(b(\psi x))   \, . \]
This requires scaling of $a(x), b(x)$ and an inverse scaling of $c(\psi x)$, where the latter could be 
combined with the scaling by $N$ in the final step of the INTT.

\subsection{Optimizing the NTT}

To optimize the NTT, it is possible to absorb the scaling by $\psi$ and also to work around 
the BitReverse as is done in \cite{longa}.\todo{add citation}  In the forward NTT, the function will return a $\psi$-scaled NTT
in bit-reverse order, which will be undone in the INTT.

Let $NTT_{sb}$ denote the function which computes the NTT of the scaled polynomial $a(\psi x)$ and where the output 
is given in bit-reversed order, in particular the output is given by 
\[  BitReverse(  [a(\psi \omega^{k}) : k \in [0 \ldots N-1]]   ) \, .  \]
The resulting algorithm is given in Algorithm~\ref{algo:ntt_sb} and $\Psi_{rev}$ is the 
array given by 
\[ \Psi_{rev} = BitReverse( [ a(\psi \omega^{k}) : k \in [0 \ldots N-1]] ) \, . \]

\begin{algorithm}[!t] \label{algo:ntt_sb}
\begin{scriptsize}
\caption{\emph{$NTT_{sb}$}}
\KwIn{A vector $a = [a[0], \ldots, a[N-1]]$ of elements in $\Z_q$ in standard order 
and $2N$-th primitive root $\psi \in \mathbb{Z}_q$ of unity, and 
precomputed table $\Psi_{rev}$ containing the powers of $\psi$ in bit-reversed order}
\KwOut{$NTT_{sb}$($a$), i.e.\ bit-reversed NTT of scaled $a(\psi x)$}
\Begin{
\everypar={\nl}
$k = 0$

\For{$len = N/2; len > 0; len = len/2$} 
{
 \For{$start = 0; start < N; start = j + len$} 
{
      $S = \Psi_{rev}[++k]$

\For{$j = start; j < start + len; ++j$} 
{
        $t = S * a[j + len] \bmod q$

        $a[j + len] = a[j] - t \bmod q$

        $a[j] = a[j] + t \bmod q$
      }
    }
  }
}
\end{scriptsize}
\end{algorithm} 

Let $INTT_{sb}$ denote the function which computes the inverse NTT of a bit-reversed array 
including scaling by $\psi^{-1}$, i.e.\ $INTT_{sb}$ satisfies
\[ INTT_{sb} (NTT_{sb}(a(x)) = a(x) \, . \]

\begin{algorithm}[!t] \label{algo:intt_sb}
\begin{scriptsize}
\caption{\emph{$INTT_{sb}$}}
\KwIn{A vector $a = [a[0], \ldots, a[N-1]]$ of elements in $\Z_q$ in bit-reversed order 
and $2N$-th primitive root $\psi \in \mathbb{Z}_q$ of unity, and 
precomputed table $\Psi_{rev}$ containing the powers of $\psi$ in bit-reversed order}
\KwOut{$INTT_{sb}$($a$) in standard ordering}
\Begin{
\everypar={\nl}
$k = N$

\For{$len = 1; len < N; len = 2*len $} 
{
   \For{$start = 0; start < N; start = j + len$} 
{
      $S = -\Psi_{rev}[--k]$

\For{$j = start; j < start + len; ++j$} 
{
        $t = a[j]$

        $a[j] = t + a[j + len] \bmod q$

        $a[j + len] = S*(t - a[j + len]) \bmod q$
        }
    }
  }

  \For{$j = 0; j < N; ++j$} 
 {
    $a[j] =a[j]/N$
  }

}
\end{scriptsize}
\end{algorithm} 

To compute the product $c(x)$ of two polynomials in $a(x), b(x) \in \Z_q[x]/(x^N+1)$, we can now simply it as 
\[  c(x) = INTT_{sb}(NTT_{sb}(a(x)) \cdot NTT_{sb}(b(x)) \, . \]

The function $NTT_{sb}$ uses $(N/2) \log_2(N)$ multiplications modulo $q$ and a total of $N \log_2(N)$ additions/subtractions modulo $q$.
Due to the final scaling, the function $INTT_{sb}$ requires $(N/2) \log_2(N) + N$ multiplications modulo $q$ and 
$N \log_2(N)$ additions/subtractions modulo $q$.

A polynomial product in $\Z_q[x]/(x^N+1)$ therefore requires $(3N/2) \log_2(N) + 2N$ multiplications and $3 N \log_2(N)$ 
additions/subtractions.  This should be compared to the $N^2$ multiplications and additions required for schoolbook multiplication.

Example: for $N = 256$ we thus require 3328 multiplications modulo $q$ and 6144 additions modulo $q$ compared to the 
65536 multiplications/additions for the schoolbook approach.

\subsection{NTT for $4$-element word arrays}

In Miden VM, a word consists of $4$ field elements and the RAM is addressable by words, so we can read or write 4 field
elements in one cycle.  As such, it is interesting to develop an NTT routine which takes into account the word-addressable
memory.   We see two possibilities depending on the application:

\subsubsection{4 parallel NTTs}

In many cases, e.g.\ in Dilithium, more than one NTT is required to be executed at the same time, e.g.\
in Dilithium, we require $k + \ell$ NTTs and $k$ INTTs for $(k,\ell) = (4,4), (6,5), (8,7)$, so we can 
easily execute $4$ NTTs in parallel.

The idea is simply to take the 4 input $N$-vectors and pack them columwise in the $4$ element words, i.e.\
the $N$ words consist of 
\[ [A[k], B[k], C[k], D[k]] \]
where $A,B,C,D$ are the $4$ input vectors.    Algorithm~\ref{algo:ntt_sb} remains exactly the same, with the assumption that
all operations are now executed in a SIMD fashion.

\textbf{Example}: For $N = 256$ this would result in $2048$ reads and $2048$ writes of a word, $2048$ SIMD-adds and $1024$ SIMD-muls.

\subsubsection{Single NTT for $4$-element word arrays}

Inspired by the $2$-element word memory in~\cite{}\todo{insert citation}, we extend this method to $4$ element words, by packing 
two consecutive words in the $2$-element array into one $4$-element word.

In particular, we simple store 4 consecutive elements in a logical array as a 4-element word.

Looking at the NTT however, we see that in one iteration elements of the form $a[j]$ and $a[j+len]$
are manipulated for $len = N/2, \ldots, 1$ where $len$ is divided by $2$ in each iteration.  This indicates that
the input array to the NTT should be ordered as follows:
\[ [  [A[2k], A[2k+N/2], A[2k+1], A[2k +1 + N/2]  ] \, . \]
Furthermore, since in the next iteration we still want to have that elements which are combined in 
the NTT butterfly are in the same word, we will have to read out a second word which contains the
elements $len/2$ removed.  This allows us to keep elements that are combined in the butterfly in 
the same word, i.e.\ in every iteration the elements saved will be of the form
\[ [  [A[2k], A[2k+len], A[2k+1], A[2k +1 + len]  ] \, . \]

The result is summarized in Algorithm~\ref{algo:ntt4_sb}.  Note that the input has to be formatted as above, 
but that the output is a linearly organized array of quadruples of the bitreversed result, i.e.\ if 
$B$ would be the output of $NTT_{sb}$, then the output of $NTT4_{sb}$ simply is 
\[ [B[4k], B[4k+1], B[4k+2], B[4k+3]] \, . \]

A similar approach can be taken for the inverse NTT, see e.g.\ \texttt{ntt\_4\_256.py} in the Code subdirectory.

\textbf{Example}: We obtain the following numbers for the NTT, where adds and muls refer to single adds and muls (not SIMD versions).
The inverse NTT requires an extra $N$ Reads and Writes and Muls if the scaling by $N$ is not folded into the final iteration
of the main loop.
\begin{center}
\begin{tabular}{|c|c|c|c|c|} 
\hline
N & Reads & Writes &  Muls & Adds \\
\hline
256 & 512 & 512 & 1024 & 2048   \\
512 & 1152 & 1152 & 2304  & 4608 \\
1024 & 2560 & 2560 &  5120 & 10240 \\
\hline
\end{tabular}
\end{center}

A savings of $N/4$ reads and writes can be obtained by combining the $len=2$ and $len=1$ loops in
Algorithm~\ref{algo:ntt4_sb}.
If a single butterfly is implemented in 1 Miden cycle, then the total number of cycles equals Reads + Writes + Muls, 
since there are Muls butterflies, which also equals the number of Adds.


\begin{algorithm}[!t] \label{algo:ntt4_sb}
\begin{scriptsize}
\caption{\emph{$NTT4_{sb}$}}
\KwIn{A vector of quadruples $a = [[A[2k], A[2k+N/2], A[2k+1], A[2k +1 + N/2] ]$ for $k =0, \ldots, N/4$ 
of elements in $\Z_q$ and $2N$-th primitive root $\psi \in \mathbb{Z}_q$ of unity, and 
precomputed table $\Psi_{rev}$ containing the powers of $\psi$ in bit-reversed order}
\KwOut{$NTT_{sb}$($a$), i.e.\ bit-reversed NTT of scaled $a(\psi x)$ in linear order
as a vector of quadruples}
\Begin{
\everypar={\nl}
$k = 0$

\For{$len = N/2; len > 2; len = len/2$} 
{
 \For{$start = 0; start < N/4; start = j + len/2$} 
{
      $S = \Psi_{rev}[++k]$

\For{$j = start; j < start + len/4; ++j$} 
{
	$A1 = a[j]$  // first quadruple
	
	$A2 = a[j+len/4]$  // second quadruple

	$t1 = S * A1[1]$

	$t2 = S * A2[1]$

	$t3 = S * A1[3]$

	$t4 = S * A2[3]$
	
	$a[j] = [A1[0] + t1, A2[0] + t2, A1[2] + t3, A2[2] + t4]$

        $a[j+ len/4] := [A1[0] - t1, A2[0] - t2, A1[2] - t3, A2[2] - t4] $

      }
    }
  }

// len = 2 done separately

\For{$j = 0; j < N/4; ++j$}
{ 
      $S =  \Psi_{rev}[++k]$

      $A1 = a[j]$

      $t1 = S * A1[1]$

      $t2 = S * A1[3]$

      $a[j] := [A1[0] + t1, A1[2] + t2, A1[0] - t1, A1[2] - t2]$
}

// len = 1 done separately

\For{$j = 0; j < N/4; ++j$}
{ 
      $S =  \Psi_{rev}[++k]$

      $A1 = a[j]$

      $t1 = S * A1[1]$

      $S =  \Psi_{rev}[++k]$

      $t2 = S * A1[3]$

      $a[j] := [A1[0] + t1, A1[2] + t2, A1[0] - t1, A1[2] - t2]$
}

}
\end{scriptsize}
\end{algorithm} 

\begin{comment}
\subsection{Arithmetic in cyclotomic rings defined by $\Phi_{3^2 \cdot 2^k}$}

To accomodate all parameter sets for the IMLWE based systems akin to Threebears, we will require arithmetic 
in $R_{p, 9, n} = \Z_q[x] / (x^{3n} - x^{3n/2} + 1)$ where $n = 2^k$ and $p = 2^64 - 2^32 + 1$ is the Miden native modulus.  
In particular, we will be interested  in the case $n = 64$ and $n = 128$.

By definition, $\Phi_{3^2 \cdot 2^k} = x^{3 2^k}  - x^{3 2^{k-1}} + 1$, has as its roots the primitive $9 n$-th roots of unity, 
of which there are $\varphi(9 n) = 3 n$.  However, for our choice of $p$, we only have primitive $3 n$-th roots of unity, since $3 | p-1$, 
but not $9 | p-1$, of which there are $n$.

Let $\zeta_3 = 4294967295$, and denote with $\zeta_{n}$ a fixed primitive $2^k$ root of unity (which will be specified later
for $n = 64$ we take $\zeta_n = 8$ and for $n = 128$ we take $\zeta_n ...$), then we can order all primitive $3 n$-th roots of unity as 
\[   [\zeta_3 \zeta_{n}^{2j+1}]_{j = 0, \ldots, n/2-1} \cup [ \zeta_3^{-1} \zeta_{n}^{2j+1}]_{j = 0, \ldots, n/2-1}  \, . \]

The arithmetic in $R_{p, 9, n}$ decomposes using CRT into the product of rings
\[  R_{p, 9, n} = \prod_{i}^{n} (\Z_q[x] / (x^3 - r_i))  \,  \]
where $r_i$ is the $i$-th primitive $3 n$-th root of unity according to the above ordering.
To compute this representation, it suffices to compute $g(x) \bmod (x^3 - r_i)$ for all $r_i$, 
where $g(x)$ is a polynomial of degree $< 3n$.  Write $g(x) = \sum_{j = 0}^{3n-1} g_i x^i$, 
and decompose $g(x)$ as 
\[  g(x) = G_0(x^3) + x G_1(x^3) + x^2 G_2(x^3) , \]
where the $G_i(x)$ are polynomials of degree $<n$, 
then it is clear that 
\[ g(x) \bmod (x^3 - r_i) = G_0(r_i) + x G_1(r_i) + x^2 G_2(r_i) \, ,  \]
so the mapping to the product representation corresponds to $3$ 
evaluations of degree $<n$ polynomials in the $3n$-th primitive roots of unity.

Let $G(x)$ denote such polynomial, then to evaluate $G(x)$ in $\zeta_3^{\pm} \zeta_n^{2j+1}$ say, 
it suffices to evaluate $H(x)_{\pm} = G(\zeta_3^\pm x)$ in the $n/2$ primitive $n$-th roots of unity for which we 
can use the NTT derived in the previous sections.  Computing $H(x)$ requires $2/3n$ multiplications.
Note that $H(x)$ has degree $<n$, but we only evaluate it in primitive $n$-th roots of unity, so we are 
essentially looking at $H(x) \bmod (x^{n/2} + 1)$.  So by folding $H(x)$ in half, we thus have 
reduced the problem to evaluation a degree $n/2$ polynomial in the primitive $n$-th roots of unity.

So overall, mapping a polynomial $g(x)$ to the product representation requires:
\bit
\item splitting $g(x)$ in 3 parts
\item for each part, computing $H_1(x) = g(\zeta_3 x) \bmod (x^n/2+1)$ and $H_2(x) = g(\zeta_3^2 x) \bmod (x^n/2 + 1)$ 
which requires $4/3n$ multipilications and $n$ additions
\item for each degree $n/2$ polynomial (there are 6 of them in total), perform an NTT costing $n/4 \log_2(n/2)$ multiplications and $n/2 \log_2(n/2)$
additions
\eit
resulting in a total of 
\[  4n + 6n/4 \log_2(n/2) \text{ multipications  and } 3 n (\log_2(n/2 + 1)) \text{ additions } \, .   \]

To derive the inverse transformation, it clearly suffices to recover $G(x)$ from its evaluations
$G(r_i)$ in the primitive $3n$-th roots of unity.  Recall that these come in two sets, 
one corresponding to $\zeta_3 \zeta_n^{2j+1}$ and one corresponding to $\zeta_3^{-1} \zeta_n^{2j+1}$.
Using the inverse NTT we can recover the polynomials 
\[ H(x)_+ = G(\zeta_3 x) \bmod (x^{n/2} + 1) \text{ and } H(x)_- = G(\zeta_3^{-1} x) \bmod (x^{n/2} + 1) \, .  \]
Note that in terms of the original coefficients $G_k$, the $j$-th coefficient of $H(x)_+$ and $H(x)_-$ are given by
\[  H_{+,j} = G_j \zeta_3^j - G_{n/2+j} \zeta_3^{(n/2+j)} \quad \text{and} \quad   H_{-,j} = G_j \zeta_3^{-j} - G_{(n/2+j)} \zeta_3^{-(n/2+j)}  \, , \]
so we can recover 
\[
G_j = \zeta_3^{-j} (1 - \zeta_3^{n})^{-1} \cdot  (H_{+, j} - \zeta_3^{(n+2j)} H_{_,j}) \quad 
G_{n/2 + j} = \zeta_3^{n/2+j} (\zeta_3^{-j} G_j - H_{-,j})  \, . 
\]
The total cost of the inverse transformation therefore is:
\bit
\item 6 inverse NTTs of length $n/2$ costing $n/4 \log_2(n/2) + n/2$ multiplications and $n/2 \log_2(n/2)$ additions
\item 3 recombinations to derive the polynomials $G_0, G_1, G_2$, each recombination costing
\[  2n \text{ multiplications  and } n \text{ additions }   , \]
\eit
for a total cost of 
\[   3n/2 \log_2(n/2) + 9n  \text{ multiplications and } 3n (\log_2(n/2) + 1) \text{ additions } \, .\]

Example: the above costs should be compared with an embedding strategy where the polynomial 
is embedded in the ring modulo $x^{4n}+1$, which allows to recover the product exactly as well.
For $n = 64$ we get for the native forward mapping above a total of 732 multiplications vs.\ 2304
for the embedding in the degree 512 ring.  For the inverse mapping we obtain 1056 multiplications vs. \ 
2816 mulitplications for the embedding strategy.

To use the above transformations, we are left with multiplication in the rings $\Z_q[x]/(x^3 - r_i)$.  Given
two elements $a(x) = a_0 + a_1 x + a_2 x^2$ and $b(x) = b_0 + b_1 x + b_2 x^2$, we could compute 
the product using Toom-Cook in 5 multiplications, but the interpolation step requires division by $6$
which would amount to a multiplication or to a seperate routine executing division by 3 and a shift.
To avoid this, we use a two step Karatsuba multiply:
\bit
\item Compute $c_0 = a_0b_0$, $c_1 = a_1 b_1$, $c_2 = a_2 b_2$
\item Compute $d_0 = (a_0 + a_1) (b_0 + b_1)$ and $d_1 = (a_1 + a_2)(b_1 + b_2)$ and $d_2 = (a_0 + a_1 + a_2)(b_0 + b_1 + b_2)$
\eit
The the product polynomial is given by:
\[
[c_0,  d_0 - c_0 - c_1,   d_2 - d_0 - d_1 + 2c_1,    d_1 - c_1 - c_2,   c_2] \, .
\]
These formulae need 6 multiplications and 14 additions to compute the product.  
Of course, we still have to reduce mod
$x^3 - r_i$, which gives another 2 multiplications and 2 additions, so a total of 8 multiplications and 16 additions.

To multiply two elements in the ring $R_{p, 9, n}$ we therefore require:
\bit
\item Two mappings to the product ring, which constists of $n$ rings of degree $3$
\item $n$ multiplications of degree 2 elements in such rings, totalling $8n$ mults and $16n$ additions
\item One inverse mapping to the polynomial ring
\eit
The overall cost therefore is:
\[   25n + 9n/2 \log_2(n/2) \text{ multiplications and }    6n \log_2(n/2) + 22 n \text{ additions } \, .   \]
Using the embedding strategy in the ring of dimension $4n$ would give
\[  64n + 12n \log_2(n/2) \text{ multiplications and }  12n \log_2(n/2) + 36n \text{ additions } \, . \]

Example: for $n = 64$ we thus obtain 3032 multiplications vs.\ 7936 multiplications for the embedding strategy.
\end{comment}

\subsection{Native parameter sets}
\label{section:native-parameter-sets}

For the public key encryption and publicly re-randomizable commitment schemes we rely on arithmetic in a ring that is natively compatible with the prime $p = 2^{64} - 2^{32} + 1$, meaning that there is no need for additional modular reduction. This ring, which is inspired by the ThreeBears NIST standardization candidate~\cite{threebears}, is defined as 
$$ R_p = \frac{\mathbb{Z}_p[X]}{(X^n+1)} $$
but embeds a lattice of dimension $4n$ through the packing scheme
$$ \mathbb{Z}_p = \frac{\mathbb{Z}[Y]}{(Y^4 - Y^2 + 1, Y-2^{16})} \enspace .$$

Multiplication in this ring can be achieved using NTTs of dimension $n$. Elements of this ring are short if the balanced expansion base $2^{16}$ of its coefficients are short. Specifically, if $a, b, c, d$ are small  (in absolute value) integers, then $a + b \cdot 2^{16} + c \cdot 2^{32} + d \cdot 2^{48}$ is a small element of $\mathbb{Z}_p$, and when the all the coefficients of a polynomial $f(X) \in R_p$ have this form then it is short.

More concretely, we use the centered binomial distribution $\Xi$ with bounds $-8$ and $8$ (both inclusive) and standard deviation $\sigma = 2$. However, $\Xi$ is a distribution of small integers; we also need field elements whose 16-bit chunks are distributed according to $\Xi$. To this end, define the distribution $\Upsilon = \sum_{i=0}^{3} \xi_i 2^{16i}$ with $\xi_i \sim \Xi$. Algorithms~\ref{algo:sample_small_integer} and~\ref{algo:sample_short_field_element} specify one way to sample these distributions. Sampling short polynomials consists of sampling a coefficient vector of length 64 whose elements are distributed according to $\Upsilon$.


\begin{algorithm}[!t] \label{algo:sample_small_integer}
\begin{scriptsize}
\caption{\emph{$\mathsf{sample\_small\_integer}$}}
\KwIn{}
\KwOut{an integer distributed according to $\Xi$}
\Begin{
\everypar={\nl}
$\mathit{bits} \xleftarrow{\$} \mathbb{Z}/2^{16}\mathbb{Z}$ // 16 random bits

$\mathit{num\_set} \leftarrow 0$

\While{$\mathit{bits} \neq 0$}{
  $\mathit{bits} \leftarrow \mathit{bits} \mathtt{\&} (\mathit{bits} - 1)$ // bitwise and
  
  $\mathit{num\_set} \leftarrow \mathit{num\_set} + 1$
}

\textbf{return} {$\mathit{num\_set} - 8$}
}
\end{scriptsize}
\end{algorithm} 

\begin{algorithm}[!t] \label{algo:sample_short_field_element}
\begin{scriptsize}
\caption{\emph{$\mathsf{sample\_short\_field\_element}$}}
\KwIn{}
\KwOut{a field element distributed according to $\Upsilon$}
\Begin{
\everypar={\nl}
$\mathit{acc} \leftarrow 0$ 

\For{$i \in \{0, \ldots, 3\}$}{
  $\xi \leftarrow \mathsf{sample\_small\_integer}()$
  
  $\mathit{acc} \leftarrow \mathit{acc} \cdot 2^{16} + \xi$
}

\textbf{return} {$\mathit{acc}$}
}
\end{scriptsize}
\end{algorithm} 

On top of the ring structure we use the module approach with dimension $m$. So specifically, there is an $m \times m$ matrix, along with $m \times 1$ and $1 \times m$ vectors whose elements belong to $R_p$. The parameter $m$ depends on the security level.

\begin{center}
\begin{tabular}{c|c}
sec. lvl & $m$ \\ \hline
$128$ & $3$ \\
$192$ & $4$ \\
$256$ & $6$
\end{tabular}
\end{center}



