%!TeX root=miden_lattices.tex
\section{Fast arithmetic in cyclotomic rings}


\subsubsection{Parameter sets}

The lattice based schemes Kyber, Saber, Falcon and Dilithium, which are all finalists in the NIST pqcrypto standardization
effort (and many other schemes), all rely on arithmetic in cyclotomic rings of the form
\[  R_{q,n} = \Z_q[x]/(x^n + 1)    \]
for some modulus $q$ (not necessarily prime) and $n = 2^k$, and more in particular $n = 256, 512, 1024$.
The moduli used by the different schemes are as follows:
\bit
\item Kyber: $q = 3329$
\item Saber: $q = 2^{13}$
\item Falcon: $q = 12289$
\item Dilithium: $q = 2^{23} - 2^{13} + 1$
\eit

When the modulus $q$ is chosen such that $2n | \varphi(q)$ (in many cases $q$ is a prime so then $2n | q-1$), 
it is well known that arithmetic in $R_q$ can be computed efficiently using the number theoretic transform.
This can be done natively for Falcon and Dilithium, almost natively for Kyber and with a work-around for Saber.

In Miden, the native modulus is $p = 2^{64} - 2^{32} + 1$, so $\Z_q$ in this case contains a root of unity 
of order $2^{32}$ and in particular of order $2^k$ for any $k = 1, \ldots, 32$, and we therefore get native
arithmetic in $R_{p,n}$ for all such $n = 2^k$.

\subsubsection{Bound on the size of $q$}

Since the $q$ used in the above schemes is different from the native $p$, we first need to give a bound for
the maximum modulus size $q$ for each ring $R_{q,n}$ such that we can recover the product exactly (using 
an extra modular reduction) from the product in $R_{p,n}$.

So assume we are given two elements  $a(x), b(x) \in R_{q,n}$ written as $a(x) = \sum_{i = 0}^{n-1} a_i x^i$
and $b(x) = \sum_{i = 0}^{n-1} b_i x^i$, then the product $c(x) = \sum_{i = 0}^{n-1} c_i x^i$ satisfies
\[  c_i = \sum_{j = 0}^{i} a_{i -j} b_j  - \sum_{j = i+1}^{n-1} a_{n-j+i} b_{j}  \, . \]
In particular, we have a sum of $n$ products of elements
in $\Z_q$ (with $\pm$), so we see that the maximum bound on $c_i$ is given 
\[ |c_i| < n q^2 \, , \]
assuming that all coefficients were represented in $[0,q)$.  If a symmetric interval $[q/2, q/2)$ was used to 
represent elements in $R_{q, n}$ the bound is clearly sufficient as well.  
Since we need to be able to recover this as an integer (it can be negative) to be able to reduce modulo $q$ afterwards,
it suffices that $p \geq 2 n q^2$.  For popular choices of $n$ above we therefore obtain the following upper bounds:
\begin{center}
\begin{tabular}{|c|c|}
\hline
$n$ & $\max \log_2(q)$ \\
\hline
$256$ &  27.5 \\
$512$ &   27 \\
$1024$ &   26.5 \\
\hline
\end{tabular}
\end{center}

\subsubsection{Lazy reduction} In the schemes that use a module structure such as Kyber, Saber and Dilithium, 
one often has to compute 
a matrix vector product $\bA \cdot \bv$ where the matrix and vector contain elements of $R_{q,n}$.
Assuming that the matrix has $\ell$ columns, we therefore could add $\ell$ such products together before
doing the final reduction modulo $q$.  The addition of $\ell$ such products simply introduces an extra 
factor of $\ell$ in the above bound.  The largest number of columns appearing in all of the above schemes
is $7$ in the case of Dilithium level 5 parameter set.  It is easy to verify that for this case we have
$p > 7 \cdot 2 \cdot 256 \cdot q^2$, so we can indeed postpone the final reduction after the addition 
of the $\ell$ products.
