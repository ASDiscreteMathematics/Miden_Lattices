\documentclass[oribibl,a4paper,envcountsame]{llncs}
\setlength\overfullrule{2mm}

\usepackage{amsmath,amssymb,todonotes,comment}
\usepackage{slashbox}
\usepackage[utf8]{inputenc}
\usepackage{tikz}

\usepackage{xcolor,colortbl}
\definecolor{linkcolor}{rgb}{0.65,0,0}
\definecolor{citecolor}{rgb}{0,0.65,0}
\definecolor{urlcolor}{rgb}{0,0,0.65}
\usepackage[colorlinks=true, linkcolor=linkcolor, urlcolor=urlcolor, citecolor=citecolor]{hyperref}

\usepackage[british]{babel}
\addto{\captionsbritish}{\renewcommand\abstractname{Abstract.}}

\usepackage{listings}

\definecolor{mGreen}{rgb}{0,0.6,0}
\definecolor{mGray}{rgb}{0.5,0.5,0.5}
\definecolor{mPurple}{rgb}{0.58,0,0.82}
% \definecolor{backgroundColour}{rgb}{0.95,0.95,0.92}
\definecolor{backgroundColour}{rgb}{0.99,0.99,0.95}

\lstdefinestyle{CStyle}{
    backgroundcolor=\color{backgroundColour},   
    commentstyle=\color{mGreen},
    keywordstyle=\color{magenta},
    numberstyle=\tiny\color{mGray},
    stringstyle=\color{mPurple},
    basicstyle=\ttfamily\footnotesize,
    breakatwhitespace=false,         
    breaklines=true,                 
    captionpos=b,                    
    keepspaces=true,                 
    numbers=left,                    
    numbersep=5pt,                  
    showspaces=false,                
    showstringspaces=false,
    showtabs=false,                  
    tabsize=2,
    language=C
}


\hyphenation{ASIA-CRYPT an-ni-hi-lat-ed non-zero mo-du-lo mo-du-lus}

\spnewtheorem*{lemma*}{Lemma}{\normalfont\bfseries}{\itshape}
\spnewtheorem*{theorem*}{Theorem}{\normalfont\bfseries}{\itshape}
\spnewtheorem*{remark*}{Remark}{\normalfont\itshape}{\normalfont}
\spnewtheorem*{conjecture*}{Conjecture}{\normalfont\bfseries}{\itshape}
\spnewtheorem*{discussion*}{Discussion}{\normalfont\itshape}{\normalfont}
\spnewtheorem*{example*}{Example}{\normalfont\itshape}{\normalfont}


\usepackage{algorithmic}
\usepackage{epsfig}
\usepackage{amssymb}
\usepackage{amsmath}
\usepackage{multicol}
\usepackage{subfigure}
\usepackage[bottom]{footmisc}
\usepackage[ruled]{algorithm2e}


\pagestyle{plain}

\newcommand{\F}{\mathbb{F}}
\newcommand{\Fpbar}{\overline{\mathbb{F}}_p}
\newcommand{\Fqbar}{\overline{\mathbb{F}}_q}
\newcommand{\Gal}{\mathrm{Gal}}
\newcommand{\N}{\mathbb{N}}
\newcommand{\OO}{\mathcal{O}}
\newcommand{\EE}{\mathcal{E}}
\newcommand\OOu[1]{\OO_{\!{#1}}}
\newcommand{\QQ}{\mathcal{Q}}
\newcommand{\RR}{\mathcal{R}}
\newcommand{\II}{\mathcal{I}}
\newcommand{\JJ}{\mathcal{J}}
\newcommand{\mA}{\mathcal{A}}
\newcommand{\Q}{\mathbb{Q}}
\newcommand{\R}{\mathbb{R}}
\newcommand{\Z}{\mathbb{Z}}
\newcommand{\C}{\mathbb{C}}
\newcommand{\Ell}{\mathcal{E}\hspace{-0.065cm}\ell\hspace{-0.035cm}\ell}
\renewcommand{\i}{\mathbf{i}}
\renewcommand{\j}{\mathbf{j}}
\renewcommand{\ij}{\mathbf{ij}}
\newcommand{\ji}{\mathbf{ji}}

\newcommand\Zsqrt[1]{\Z[\hspace{-.1em}\sqrt{#1}]}
\newcommand\Qsqrt[1]{\Q(\!\sqrt{#1})}
\newcommand\Leg[2]{\left( \frac{#1}{#2}\right)}
\newcommand{\cl}{\operatorname{cl}}
\newcommand{\Aut}{\operatorname{Aut}}
\newcommand{\End}{\mathrm{End}}
\newcommand{\Endp}{\mathrm{End}_p}
\newcommand{\id}{\mathrm{id}}

\newcommand{\bit}{\begin{itemize}}
\newcommand{\eit}{\end{itemize}}

\newcommand{\bA}{\mathbf{A}}
\newcommand{\bs}{\mathbf{s}}
\newcommand{\bv}{\mathbf{v}}
\newcommand{\bz}{\mathbf{z}}
\newcommand{\bh}{\mathbf{h}}
\newcommand{\bt}{\mathbf{t}}
\newcommand{\by}{\mathbf{y}}
\newcommand{\bw}{\mathbf{w}}

\newcommand{\afrak}{\mathfrak{a}}
\newcommand{\bfrak}{\mathfrak{b}}
\newcommand{\cfrak}{\mathfrak{c}}
\newcommand{\gfrak}{\mathfrak{g}}
\newcommand{\kfrak}{\mathfrak{k}}
\newcommand{\lfrak}{\mathfrak{l}}
\newcommand{\mfrak}{\mathfrak{m}}
\newcommand{\pfrak}{\mathfrak{p}}
\newcommand{\rfrak}{\mathfrak{r}}
\newcommand{\tfrak}{\mathfrak{t}}
\newcommand{\ufrak}{\mathfrak{u}}
\newcommand{\xfrak}{\mathfrak{x}}
\newcommand{\yfrak}{\mathfrak{y}}
\newcommand{\zfrak}{\mathfrak{z}}
\newcommand{\sfrak}{\mathfrak{s}}
\newcommand{\Pfrak}{\mathfrak{P}}
\newcommand{\tm}{T_m}
\newcommand{\Pinf}{\mathbf{0}}

\newcommand\norm[1]{\mathrm{N}(#1)}
\newcommand\trace{\operatorname{tr}}

\newcommand\im{\operatorname{im}}
\newcommand\Frob{\operatorname{Frob}}

\newcommand{\hooklongrightarrow}{\lhook\joinrel\longrightarrow}
\newcommand{\hooklongleftarrow}{\longleftarrow\joinrel\rhook}
\DeclareRobustCommand\twoheadlongrightarrow{\relbar\joinrel\twoheadrightarrow}

\newcommand{\infpt}{\infty} % point at infinity
\DeclareMathOperator{\poly}{poly}
\DeclareMathOperator{\ord}{ord}
\DeclareMathOperator{\val}{val}
\DeclareMathOperator{\cond}{cond}
\DeclareMathOperator{\ar}{ar}

\newcommand\red{\textcolor{red}}

\newif\ifsubmission
%\submissiontrue
\submissionfalse

\title{Lattice-Based Cryptography in Miden VM}
\ifsubmission
\author{\vspace*{-1cm} }
\institute{\vspace*{-1cm}\ }
\else
\author{
    Alan Szepieniec\inst{1}
    \and
    Frederik Vercauteren\inst{2}
    \\[2mm]
    {\footnotesize
        \email{alan@asdm.gmbh},
        \email{frederik.vercauteren@kuleuven.be}
    }
}
\institute{
     AS Discrete Mathematics GmbH, Switzerland
     \and 
     imec-COSIC, KU Leuven, Belgium
}
\fi

\date{\today}

\begin{document}

\maketitle

\ifsubmission
\else
\vspace{-1ex}
\begingroup
  \makeatletter
  \def\@thefnmark{$\ast$}\relax
  \@footnotetext{\relax
  Date of this document: \today.}
\endgroup
\vspace{-1ex}
\fi

\begin{abstract}

\end{abstract}

\begin{keywords}

\end{keywords}

%%%%%%%%%%%%%%%%%%%%%%%%%%%%%%%%%%%%%%%%%%%%%%%%%%%%%%%%%%%%%%%%%%%%%%%%%%%%%%%%

%!TeX root=miden_lattices.tex
\section{Introduction}
 

%!TeX root=miden_lattices.tex
\section{Fast arithmetic in cyclotomic rings}

\subsection{Parameter sets}

The lattice based schemes Kyber, Saber, Falcon and Dilithium, which are all finalists in the NIST pqcrypto standardization
effort (and many other schemes), all rely on arithmetic in cyclotomic rings of the form
\[  R_{q,n} = \Z_q[x]/(x^n + 1)    \]
for some modulus $q$ (not necessarily prime) and $n = 2^k$, and more in particular $n = 256, 512, 1024$.
The moduli used by the different schemes are as follows:
\bit
\item Kyber: $q = 3329$
\item Saber: $q = 2^{13}$
\item Falcon: $q = 12289$
\item Dilithium: $q = 2^{23} - 2^{13} + 1$
\eit

When the modulus $q$ is chosen such that $2n | \varphi(q)$ (in many cases $q$ is a prime so then $2n | q-1$), 
it is well known that arithmetic in $R_q$ can be computed efficiently using the number theoretic transform.
This can be done natively for Falcon and Dilithium, almost natively for Kyber and with a work-around for Saber.

In Miden, the native modulus is $p = 2^{64} - 2^{32} + 1$, so $\Z_q$ in this case contains a root of unity 
of order $2^{32}$ and in particular of order $2^k$ for any $k = 1, \ldots, 32$, and we therefore get native
arithmetic in $R_{p,n}$ for all such $n = 2^k$.

\subsection{Bound on the size of $q$}

Since the $q$ used in the above schemes is different from the native $p$, we first need to give a bound for
the maximum modulus size $q$ for each ring $R_{q,n}$ such that we can recover the product exactly (using 
an extra modular reduction) from the product in $R_{p,n}$.

So assume we are given two elements  $a(x), b(x) \in R_{q,n}$ written as $a(x) = \sum_{i = 0}^{n-1} a_i x^i$
and $b(x) = \sum_{i = 0}^{n-1} b_i x^i$, then the product $c(x) = \sum_{i = 0}^{n-1} c_i x^i$ satisfies
\[  c_i = \sum_{j = 0}^{i} a_{i -j} b_j  - \sum_{j = i+1}^{n-1} a_{n-j+i} b_{j}  \, . \]
In particular, we have a sum of $n$ products of elements
in $\Z_q$ (with $\pm$), so we see that the maximum bound on $c_i$ is given 
\[ |c_i| < n q^2 \, , \]
assuming that all coefficients were represented in $[0,q)$.  If a symmetric interval $[q/2, q/2)$ was used to 
represent elements in $R_{q, n}$ the bound is clearly sufficient as well.  
Since we need to be able to recover this as an integer (it can be negative) to be able to reduce modulo $q$ afterwards,
it suffices that $p \geq 2 n q^2$.  For popular choices of $n$ above we therefore obtain the following upper bounds:
\begin{center}
\begin{tabular}{|c|c|}
\hline
$n$ & $\max \log_2(q)$ \\
\hline
$256$ &  27.5 \\
$512$ &   27 \\
$1024$ &   26.5 \\
\hline
\end{tabular}
\end{center}

\subsection{Lazy reduction} In the schemes that use a module structure such as Kyber, Saber and Dilithium, 
one often has to compute 
a matrix vector product $\bA \cdot \bv$ where the matrix and vector contain elements of $R_{q,n}$.
Assuming that the matrix has $\ell$ columns, we therefore could add $\ell$ such products together before
doing the final reduction modulo $q$.  The addition of $\ell$ such products simply introduces an extra 
factor of $\ell$ in the above bound.  The largest number of columns appearing in all of the above schemes
is $7$ in the case of Dilithium level 5 parameter set.  It is easy to verify that for this case we have
$p > 7 \cdot 2 \cdot 256 \cdot q^2$, so we can indeed postpone the final reduction after the addition 
of the $\ell$ products.

\subsection{Multiplication using NTT}

Let $N$ be a power of $2$ and assume that $q-1 \equiv 0 \bmod 2N$, and let $\psi$ be a primitive $2N$-th root 
of unity and $\omega = \psi^2$ a primitive $N$-th root of unity.  The $N$-point NTT of a sequence 
$[a[0], \ldots, a[N-1]]$ is denoted as $\tilde{a} = NTT(a)$ and defined by $\tilde{a}[i] = \sum_{j = 0}^{N-1} a[j] \omega^{i j}$
for $i = 0, \ldots, n-1$.  The inverse transformation $b = INTT(\tilde{a})$ is given by 
$b[i] = \frac{1}{N} \sum_{j = 0}^{N-1} a[j] \omega^{-i j}$, which can be computed by replacing $\omega$ 
by $\omega^{-1}$ and scaling by $N$.

Since $\omega$ is an $N$-th root of unity, note that this also corresponds to the evaluation 
of the polynomial $a(x) \in \Z_q[x]/(x^N - 1)$ in $\omega$, and as such we can multiply two 
polynomials in $a(x), b(x) \in \Z_q[x]/(x^N - 1)$ by computing
\[  c(x) = INTT(NTT(a) \cdot NTT(b)  \]
where $\cdot$ denotes pointwise multiplication.

The standard approach to NTT is given in Algorithm~\ref{algo:ntt1} where BitReverse
computes an array $A$ such that $A[k] = a[BitReverse(k)]$ obtained by reversing
the binary expansion of $k$ using $b = \log_2(N)$ bits to write $k$.
In particular, if $k = k_0 \ldots k_{b-1}$ then $BitReverse(k) = k_{b-1} \ldots k_0$. 
\begin{algorithm}[!t] \label{algo:ntt1}
\begin{scriptsize}
\caption{\emph{Iterative NTT}}
\KwIn{Polynomial $a(x) \in \mathbb{Z}_q[\mathbf{x}]$ of
degree $N-1$ and $N$-th primitive root $\omega_N \in \mathbb{Z}_q$ of unity}
\KwOut{Polynomial $A(x) \in \mathbb{Z}_q[\mathbf{x}]$ = NTT($a$)}
\Begin{
\everypar={\nl}
$A \leftarrow BitReverse(a)$;

\For{$m=2$ to $N$ by $m=2m$}
{
$\omega_m \leftarrow \omega_N^{N/m}$ \;

$\omega \leftarrow 1$ \;

\For{$j=0$ to $m/2-1$}
{
\For{$k=0$ to $N-1$ by $m$}
{

$t \leftarrow \omega \cdot A[k+j+m/2]$ \;
$u \leftarrow A[k+j]$ \;
$A[k+j] \leftarrow u+t \bmod q$ \;
$A[k+j+m/2] \leftarrow u-t \bmod q$ \;
}
$\omega \leftarrow \omega \cdot \omega_m$ \;
}
}
}
\end{scriptsize}
\end{algorithm} 


The above however is not directly applicable since we need arithmetic in the ring $\Z_q[x]/(x^N+1)$.
Note that the roots of $x^N+1$ are given by $\psi^{2k + 1}$ for $k = 0, \ldots, n-1$, so we would 
need to compute the evalutions of $a(x)$ in those powers.  Since $\psi^{2k + 1} = \omega^k \psi$, 
so we could use the standard NTT above on the scaled polynomial $a(\psi x)$, and we would obtain
\[  c(\psi x) =    INTT(NTT(a(\psi x)) \cdot NTT(b(\psi x))   \, . \]
This requires scaling of $a(x), b(x)$ and an inverse scaling of $c(\psi x)$, where the latter could be 
combined with the scaling by $N$ in the final step of the INTT.

\subsection{Optimizing the NTT}

To optimize the NTT, it is possible to absorb the scaling by $\psi$ and also to work around 
the BitReverse as is done in \cite{longa}.  In the forward NTT, the function will return a $\psi$-scaled NTT
in bit-reverse order, which will be undone in the INTT.

Let $NTT_{sb}$ denote the function which computes the NTT of the scaled polynomial $a(\psi x)$ and where the output 
is given in bit-reversed order, in particular the output is given by 
\[  BitReverse(  [a(\psi \omega^{k}) : k \in [0 \ldots N-1]]   ) \, .  \]
The resulting algorithm is given in Algorithm~\ref{algo:ntt_sb} and $\Psi_{rev}$ is the 
array given by 
\[ \Psi_{rev} = BitReverse( [ \psi^{k}) : k \in [0 \ldots N-1]] ) \, . \]

\begin{algorithm}[!t] \label{algo:ntt_sb}
\begin{scriptsize}
\caption{\emph{$NTT_{sb}$}}
\KwIn{A vector $a = [a[0], \ldots, a[N-1]]$ of elements in $\Z_q$ in standard order 
and $2N$-th primitive root $\psi \in \mathbb{Z}_q$ of unity, and 
precomputed table $\Psi_{rev}$ containing the powers of $\psi$ in bit-reversed order}
\KwOut{$NTT_{sb}$($a$), i.e.\ bit-reversed NTT of scaled $a(\psi x)$}
\Begin{
\everypar={\nl}
$k = 0$

\For{$len = N/2; len > 0; len = len/2$} 
{
 \For{$start = 0; start < N; start = j + len$} 
{
      $S = \Psi_{rev}[++k]$

\For{$j = start; j < start + len; ++j$} 
{
        $t = S * a[j + len] \bmod q$

        $a[j + len] = a[j] - t \bmod q$

        $a[j] = a[j] + t \bmod q$
      }
    }
  }
}
\end{scriptsize}
\end{algorithm} 

Let $INTT_{sb}$ denote the function which computes the inverse NTT of a bit-reversed array 
including scaling by $\psi^-1$, i.e.\ $INTT_{sb}$ satisfies
\[ INTT_{sb} (NTT_{sb}(a(x)) = a(x) \, . \]

\begin{algorithm}[!t] \label{algo:intt_sb}
\begin{scriptsize}
\caption{\emph{$INTT_{sb}$}}
\KwIn{A vector $a = [a[0], \ldots, a[N-1]]$ of elements in $\Z_q$ in bit-reversed order 
and $2N$-th primitive root $\psi \in \mathbb{Z}_q$ of unity, and 
precomputed table $\Psi_{rev}$ containing the powers of $\psi$ in bit-reversed order}
\KwOut{$INTT_{sb}$($a$) in standard ordering}
\Begin{
\everypar={\nl}
$k = N$

\For{$len = 1; len < N; len = 2*len $} 
{
   \For{$start = 0; start < N; start = j + len$} 
{
      $S = -\Psi_{rev}[--k]$

\For{$j = start; j < start + len; ++j$} 
{
        $t = a[j]$

        $a[j] = t + a[j + len] \bmod q$

        $a[j + len] = S*(t - a[j + len]) \bmod q$
        }
    }
  }

  \For{$j = 0; j < N; ++j$} 
 {
    $a[j] =a[j]/N$
  }

}
\end{scriptsize}
\end{algorithm} 

To compute the product $c(x)$ of two polynomials in $a(x), b(x) \in \Z_q[x]/(x^N+1)$, we can now simply it as 
\[  c(x) = INTT_{sb}(NTT_{sb}(a(x)) \cdot NTT_{sb}(b(x)) \, . \]

The function $NTT_{sb}$ uses $(N/2) \log_2(N)$ multiplications modulo $q$ and a total of $N \log_2(N)$ additions/subtractions modulo $q$.
Due to the final scaling, the function $INTT_{sb}$ requires $(N/2) \log_2(N) + N$ multiplications modulo $q$ and 
$N \log_2(N)$ additions/subtractions modulo $q$.

A polynomial product in $\Z_q[x]/(x^N+1)$ therefore requires $(3N/2) \log_2(N) + 2N$ multiplications and $3 N \log_2(N)$ 
additions/subtractions.  This should be compared to the $N^2$ multiplications and additions required for schoolbook multiplication.

Example: for $N = 256$ we thus require 3328 multiplications modulo $q$ and 6144 additions modulo $q$ compared to the 
65536 multiplications/additions for the schoolbook approach.

\subsection{NTT for $4$-element word arrays}

In Miden VM, a word consists of $4$ field elements and the RAM is word-addressable so we can read/write 4 field
elements in one cycle.  As such, it is interesting to develop an NTT routine which takes into account the word-addressable
memory.   We see two possibilities depending on the application:

\subsubsection{4 parallel NTTs}

In many cases, e.g.\ in Dilithium, more than one NTT is required to be executed at the same time, e.g.\
in Dilithium, we require $k + \ell$ NTTs and $k$ INTTs for $(k,\ell) = (4,4), (6,5), (8,7)$, so we can 
easily execute $4$ NTTs in parallel.

The idea is simply to take the 4 input $N$-vectors and pack them columwise in the $4$ element words, i.e.\
the $N$ words consist of 
\[ [A[k], B[k], C[k], D[k]] \]
where $A,B,C,D$ are the $4$ input vectors.    Algorithm~\ref{algo:ntt_sb} remains exactly the same, with the assumption that
all operations are now executed in a SIMD fashion.

\textbf{Example}: For $N = 256$ this would result in $2048$ reads and $2048$ writes of a word, $2048$ SIMD-adds and $1024$ SIMD-muls.

\subsubsection{Single NTT for $4$-element word arrays}

Inspired by the $2$-element word memory in~\cite{}, we extend this method to $4$ element words, by packing 
two consecutive words in the $2$-element array into one $4$-element word.

In particular, we simple store 4 consecutive elements in a logical array as a 4-element word.

Looking at the NTT however, we see that in one iteration elements of the form $a[j]$ and $a[j+len]$
are manipulated for $len = N/2, \ldots, 1$ where $len$ is divided by $2$ in each iteration.  This indicates that
the input array to the NTT should be ordered as follows:
\[ [  [A[2k], A[2k+N/2], A[2k+1], A[2k +1 + N/2]  ] \, . \]
Furthermore, since in the next iteration we still want to have that elements which are combined in 
the NTT butterfly are in the same word, we will have to read out a second word which contains the
elements $len/2$ removed.  This allows us to keep elements that are combined in the butterfly in 
the same word, i.e.\ in every iteration the elements saved will be of the form
\[ [  [A[2k], A[2k+len], A[2k+1], A[2k +1 + len]  ] \, . \]

The result is summarized in Algorithm~\ref{algo:ntt4_sb}.  Note that the input has to be formatted as above, 
but that the output is a linearly organized array of quadruples of the bitreversed result, i.e.\ if 
$B$ would be the output of $NTT_{sb}$, then the output of $NTT4_{sb}$ simply is 
\[ [B[4k], B[4k+1], B[4k+2], B[4k+3]] \, . \]

A similar approach can be taken for the inverse NTT, see e.g.\ \texttt{ntt\_4\_256.py} in the Code subdirectory.

\textbf{Example}: We obtain the following numbers for the NTT, where adds and muls refer to single adds and muls (not SIMD versions).
The inverse NTT requires an extra $N$ Reads and Writes and Muls if the scaling by $N$ is not folded into the final iteration
of the main loop.
\begin{center}
\begin{tabular}{|c|c|c|c|c|} 
\hline
N & Reads & Writes &  Muls & Adds \\
\hline
256 & 512 & 512 & 1024 & 2048   \\
512 & 1152 & 1152 & 2304  & 4608 \\
1024 & 2560 & 2560 &  5120 & 10240 \\
\hline
\end{tabular}
\end{center}

A savings of $N/4$ reads and writes can be obtained by combining the $len=2$ and $len=1$ loops in
Algorithm~\ref{algo:ntt4_sb}.
If a single butterfly is implemented in 1 Miden cycle, then the total number of cycles equals Reads + Writes + Muls, 
since there are Muls butterflies, which also equals the number of Adds.


\begin{algorithm}[!t] \label{algo:ntt4_sb}
\begin{scriptsize}
\caption{\emph{$NTT4_{sb}$}}
\KwIn{A vector of quadruples $a = [[A[2k], A[2k+N/2], A[2k+1], A[2k +1 + N/2] ]$ for $k =0, \ldots, N/4$ 
of elements in $\Z_q$ and $2N$-th primitive root $\psi \in \mathbb{Z}_q$ of unity, and 
precomputed table $\Psi_{rev}$ containing the powers of $\psi$ in bit-reversed order}
\KwOut{$NTT_{sb}$($a$), i.e.\ bit-reversed NTT of scaled $a(\psi x)$ in linear order
as a vector of quadruples}
\Begin{
\everypar={\nl}
$k = 0$

\For{$len = N/2; len > 2; len = len/2$} 
{
 \For{$start = 0; start < N/4; start = j + len/2$} 
{
      $S = \Psi_{rev}[++k]$

\For{$j = start; j < start + len/4; ++j$} 
{
	$A1 = a[j]$  // first quadruple
	
	$A2 = a[j+len/4]$  // second quadruple

	$t1 = S * A1[1]$

	$t2 = S * A2[1]$

	$t3 = S * A1[3]$

	$t4 = S * A2[3]$
	
	$a[j] = [A1[0] + t1, A2[0] + t2, A1[2] + t3, A2[2] + t4]$

        $a[j+ len/4] := [A1[0] - t1, A2[0] - t2, A1[2] - t3, A2[2] - t4] $

      }
    }
  }

// len = 2 done separately

\For{$j = 0; j < N/4; ++j$}
{ 
      $S =  \Psi_{rev}[++k]$

      $A1 = a[j]$

      $t1 = S * A1[1]$

      $t2 = S * A1[3]$

      $a[j] := [A1[0] + t1, A1[2] + t2, A1[0] - t1, A1[2] - t2]$
}

// len = 1 done separately

\For{$j = 0; j < N/4; ++j$}
{ 
      $S =  \Psi_{rev}[++k]$

      $A1 = a[j]$

      $t1 = S * A1[1]$

      $S =  \Psi_{rev}[++k]$

      $t2 = S * A1[3]$

      $a[j] := [A1[0] + t1, A1[2] + t2, A1[0] - t1, A1[2] - t2]$
}

}
\end{scriptsize}
\end{algorithm} 

\subsection{Arithmetic in cyclotomic rings defined by $\Phi_{3^2 \cdot 2^k}$}

To accomodate all parameter sets for the IMLWE based systems akin to Threebears, we will require arithmetic 
in $R_{p, 9, n} = \Z_q[x] / (x^{3n} - x^{3n/2} + 1)$ where $n = 2^k$ and $p = 2^64 - 2^32 + 1$ is the Miden native modulus.  
In particular, we will be interested  in the case $n = 64$ and $n = 128$.

By definition, $\Phi_{3^2 \cdot 2^k} = x^{3 2^k}  - x^{3 2^{k-1}} + 1$, has as its roots the primitive $9 n$-th roots of unity, 
of which there are $\varphi(9 n) = 3 n$.  However, for our choice of $p$, we only have primitive $3 n$-th roots of unity, since $3 | p-1$, 
but not $9 | p-1$, of which there are $n$.

Let $\zeta_3 = 4294967295$, and denote with $\zeta_{n}$ a fixed primitive $2^k$ root of unity (which will be specified later
for $n = 64$ we take $\zeta_n = 8$ and for $n = 128$ we take $\zeta_n ...$), then we can order all primitive $3 n$-th roots of unity as 
\[   [\zeta_3 \zeta_{n}^{2j+1}]_{j = 0, \ldots, n/2-1} \cup [ \zeta_3^{-1} \zeta_{n}^{2j+1}]_{j = 0, \ldots, n/2-1}  \, . \]

The arithmetic in $R_{p, 9, n}$ decomposes using CRT into the product of rings
\[  R_{p, 9, n} = \prod_{i}^{n} (\Z_q[x] / (x^3 - r_i))  \,  \]
where $r_i$ is the $i$-th primitive $3 n$-th root of unity according to the above ordering.
To compute this representation, it suffices to compute $g(x) \bmod (x^3 - r_i)$ for all $r_i$, 
where $g(x)$ is a polynomial of degree $< 3n$.  Write $g(x) = \sum_{j = 0}^{3n-1} g_i x^i$, 
and decompose $g(x)$ as 
\[  g(x) = G_0(x^3) + x G_1(x^3) + x^2 G_2(x^3) , \]
where the $G_i(x)$ are polynomials of degree $<n$, 
then it is clear that 
\[ g(x) \bmod (x^3 - r_i) = G_0(r_i) + x G_1(r_i) + x^2 G_2(r_i) \, ,  \]
so the mapping to the product representation corresponds to $3$ 
evaluations of degree $<n$ polynomials in the $3n$-th primitive roots of unity.

Let $G(x)$ denote such polynomial, then to evaluate $G(x)$ in $\zeta_3^{\pm} \zeta_n^{2j+1}$ say, 
it suffices to evaluate $H(x)_{\pm} = G(\zeta_3^\pm x)$ in the $n/2$ primitive $n$-th roots of unity for which we 
can use the NTT derived in the previous sections.  Computing $H(x)$ requires $2/3n$ multiplications.
Note that $H(x)$ has degree $<n$, but we only evaluate it in primitive $n$-th roots of unity, so we are 
essentially looking at $H(x) \bmod (x^{n/2} + 1)$.  So by folding $H(x)$ in half, we thus have 
reduced the problem to evaluation a degree $n/2$ polynomial in the primitive $n$-th roots of unity.

So overall, mapping a polynomial $g(x)$ to the product representation requires:
\bit
\item splitting $g(x)$ in 3 parts
\item for each part, computing $H_1(x) = g(\zeta_3 x) \bmod (x^n/2+1)$ and $H_2(x) = g(\zeta_3^2 x) \bmod (x^n/2 + 1)$ 
which requires $4/3n$ multipilications and $n$ additions
\item for each degree $n/2$ polynomial (there are 6 of them in total), perform an NTT costing $n/4 \log_2(n/2)$ multiplications and $n/2 \log_2(n/2)$
additions
\eit
resulting in a total of 
\[  4n + 6n/4 \log_2(n/2) \text{ multipications  and } 3 n (\log_2(n/2 + 1)) \text{ additions } \, .   \]

To derive the inverse transformation, it clearly suffices to recover $G(x)$ from its evaluations
$G(r_i)$ in the primitive $3n$-th roots of unity.  Recall that these come in two sets, 
one corresponding to $\zeta_3 \zeta_n^{2j+1}$ and one corresponding to $\zeta_3^{-1} \zeta_n^{2j+1}$.
Using the inverse NTT we can recover the polynomials 
\[ H(x)_+ = G(\zeta_3 x) \bmod (x^{n/2} + 1) \text{ and } H(x)_- = G(\zeta_3^{-1} x) \bmod (x^{n/2} + 1) \, .  \]
Note that in terms of the original coefficients $G_k$, the $j$-th coefficient of $H(x)_+$ and $H(x)_-$ are given by
\[  H_{+,j} = G_j \zeta_3^j - G_{n/2+j} \zeta_3^{(n/2+j)} \quad \text{and} \quad   H_{-,j} = G_j \zeta_3^{-j} - G_{(n/2+j)} \zeta_3^{-(n/2+j)}  \, , \]
so we can recover 
\[
G_j = \zeta_3^{-j} (1 - \zeta_3^{n})^{-1} \cdot  (H_{+, j} - \zeta_3^{(n+2j)} H_{_,j}) \quad 
G_{n/2 + j} = \zeta_3^{n/2+j} (\zeta_3^{-j} G_j - H_{-,j})  \, . 
\]
The total cost of the inverse transformation therefore is:
\bit
\item 6 inverse NTTs of length $n/2$ costing $n/4 \log_2(n/2) + n/2$ multiplications and $n/2 \log_2(n/2)$ additions
\item 3 recombinations to derive the polynomials $G_0, G_1, G_2$, each recombination costing
\[  2n \text{ multiplications  and } n \text{ additions }   , \]
\eit
for a total cost of 
\[   3n/2 \log_2(n/2) + 9n  \text{ multiplications and } 3n (\log_2(n/2) + 1) \text{ additions } \, .\]

Example: the above costs should be compared with an embedding strategy where the polynomial 
is embedded in the ring modulo $x^{4n}+1$, which allows to recover the product exactly as well.
For $n = 64$ we get for the native forward mapping above a total of 732 multiplications vs.\ 2304
for the embedding in the degree 512 ring.  For the inverse mapping we obtain 1056 multiplications vs. \ 
2816 mulitplications for the embedding strategy.

To use the above transformations, we are left with multiplication in the rings $\Z_q[x]/(x^3 - r_i)$.  Given
two elements $a(x) = a_0 + a_1 x + a_2 x^2$ and $b(x) = b_0 + b_1 x + b_2 x^2$, we could compute 
the product using Toom-Cook in 5 multiplications, but the interpolation step requires division by $6$
which would amount to a multiplication or to a seperate routine executing division by 3 and a shift.
To avoid this, we use a two step Karatsuba multiply:
\bit
\item Compute $c_0 = a_0b_0$, $c_1 = a_1 b_1$, $c_2 = a_2 b_2$
\item Compute $d_0 = (a_0 + a_1) (b_0 + b_1)$ and $d_1 = (a_1 + a_2)(b_1 + b_2)$ and $d_2 = (a_0 + a_1 + a_2)(b_0 + b_1 + b_2)$
\eit
The the product polynomial is given by:
\[
[c_0,  d_0 - c_0 - c_1,   d_2 - d_0 - d_1 + 2c_1,    d_1 - c_1 - c_2,   c_2] \, .
\]
These formulae need 6 multiplications and 14 additions to compute the product.  
Of course, we still have to reduce mod
$x^3 - r_i$, which gives another 2 multiplications and 2 additions, so a total of 8 multiplications and 16 additions.

To multiply two elements in the ring $R_{p, 9, n}$ we therefore require:
\bit
\item Two mappings to the product ring, which constists of $n$ rings of degree $3$
\item $n$ multiplications of degree 2 elements in such rings, totalling $8n$ mults and $16n$ additions
\item One inverse mapping to the polynomial ring
\eit
The overall cost therefore is:
\[   25n + 9n/2 \log_2(n/2) \text{ multiplications and }    6n \log_2(n/2) + 22 n \text{ additions } \, .   \]
Using the embedding strategy in the ring of dimension $4n$ would give
\[  64n + 12n \log_2(n/2) \text{ multiplications and }  12n \log_2(n/2) + 36n \text{ additions } \, . \]

Example: for $n = 64$ we thus obtain 3032 multiplications vs.\ 7936 multiplications for the embedding strategy.












%!TeX root=miden_lattices.tex
\section{Signatures}
\label{section:signatures}
 
\subsection{Dilithium}

Dilithium is a module-LWE based signature scheme based on Fiat-Shamir with aborts.  The base ring is given 
by $R_{q,n} = \Z_q[x]/(x^n+1)$ with $q = 2^{23} - 2^{13} - 1 = 8380417$.

The public key consists of the high bits (13 lower order bits are dropped) of MLWE-samples
\[ \bt :=  \bA \bs_1 + \bs_2 \]
where $\bA$ is a $k \times \ell$  matrix over $R_{q,n}$ derived from a seed $\rho$ 
and $\bs_1, \bs_2$ vectors of length $\ell$ and $k$ make up the secret key and 
contain elements in $R_{q,n}$ with small coefficients of max size $\eta$.
A public key then contains the following data $pk = (\rho, \bt_1)$ where $\bt_1$
is obtained from $\bt$ by dropping the lowest $13$ bits.
A summary of the relevant (for verification) parameters is given in Table~\ref{tab:dilithium}.

\begin{table}\caption{Dilithium parameters used in verification}\label{tab:dilithium}
\begin{center}
\begin{tabular}{c|c|c|c}
\hline
Security level & 2 & 3 & 5 \\
\hline
$(k,\ell)$ & (4,4) & (6,5) & (8,7) \\
$\eta$ & 2 & 4 & 2 \\
$\beta$ & $78$ & $196$ & $120$ \\
$\gamma_1$ & $2^{17}$ & $2^{19}$ & $2^{19}$ \\ 
$\gamma_2$ & $(q-1)/88$ & $(q-1)/32$ & $(q-1)/32$ \\
$\omega$ & $80$ & $55$ & $75$ \\
\hline
\end{tabular}
\end{center}
\end{table}

A signature $\sigma = (\tilde{c}, \bz, \bh)$ consists of 3 components:
\bit
\item $\tilde{c}$: a challenge of $256$ bits obtained as the hash of the public key, message and $\bA \by$ where $\by$
is a vector with elements in $R_{q,n}$ with coefficients smaller than $\gamma_1$. 
\item $\bz$: a vector of length $\ell$ of polynomials with small coefficients, in particular, smaller than $\gamma_1 - \beta$.
\item $\bh$: a vector of length $k$ consisting of $k$ hint polynomials (essentially the overflows of a particular sum) whose
coefficients are ${0,1}$ and the max number of $1$'s in $\bh$ is $\omega$.
\eit

The signature verification then proceeds as follows, where $H$ is SHAKE-256 and the specification
of the other subroutines can be found in the original specification~\cite{dilithium}.
\begin{algorithm}[!ht] \label{algo:dilithium_verify}
\begin{scriptsize}
\caption{\emph{Dilithium verification}}
\KwIn{public key $pk = (\rho, \bt_1)$, message $M$, signature $\sigma = (\tilde{c}, \bz, \bh)$}
\KwOut{boolean indicating if signature is valid}
\Begin{
\everypar={\nl}
$\bA \in R_{q,n}^{k \times \ell}$ := ExpandA($\rho$)
 
$\mu \in \{0,1\}^{512} := H(H(\rho || \bt_1) || M)$

$c := SampleInBall(\tilde{c})$

$\bw_1' := UseHint_q(h, \bA \bz - 2^d \cdot c \bt_1, 2 \gamma_2)$

return $||\bz||_{\infty} < \gamma_1 - \beta$ and $\tilde{c} = H(\mu || \bw_1')$ and $\# 1's$ in $\bh$ is $\leq \omega$
}
\end{scriptsize}
\end{algorithm} 

The matrix $\bA \in R_{q,n}^{k \times \ell}$ is generated from $\rho$ but directly into the NTT domain.
In particular, every element $a_{i,j} \in R_q$ is represented in a very specific format, which is the following:
\bit
\item Let $r = 1753$ which is a $512$-th root of unity modulo $q$
\item A polynomial $a(x) \in R_q$ in NTT representation then is given by the array:
\[ [a(r_0), a(-r_0), a(r_1), a(-r_1), \ldots, a(r_{127}), a(-r_{127})] \]
\eit
where by definition $r_i = r^{brv(128 + i)}$ where $brv(k)$ denotes the 8-bit bitreversal of the number $k$.
Note that this is the same result as the bit-reversed version of the normal array
\[  [a(r), a(r^3), \ldots, a(r^{N-1})] \, .  \]


This is done to speed up step 4 in the verification procedure above, which can be computed as 
\[ NTT^{-1} (  \bA \cdot NTT(\bz)  - NTT(c) \cdot NTT(\bt_1 \cdot 2^d))  \]
which requires:
\bit
\item $\ell + k + 1$ NTTs
\item $k$ INTTs
\item $k (\ell + 1)$ pointwise multiplications of arrays of 256 elements in $\F_q$
\eit

\subsection{Falcon}

Falcon~\cite{falcon} is a lattice-based signature scheme using NTRU lattices and a hash-and-sign approach following GPV~\cite{}.
Falcon works in a cyclotomic ring $R_{q,n} =  \Z_q[x]/(x^n+1)$ with $n = 2^k$, and an NTRU public key consists 
of a polynomial $h \in R_{q,n}$ which is computed as $g \cdot f^{-1}$ in $R_{q,n}$ where $f,g \in R_{q,n}$ that have small coefficients.
Recovering $f, g$ from $h$ corresponds to the NTRU-problem.  

Falcon specifies two parameter sets $n = 512$ and $n = 1024$ corresponding to NIST security levels I and V.  The modulus 
$q = 12289 = 12 \cdot 1024 + 1$ is prime and fixed for both sets.  The coefficients of the polynomials $f,g$ are sampled
from a discrete Gaussian with standardard deviation $\sigma_{f,g} = 1.17 \sqrt{q/2n}$.

The signature of a message $m$ consists of a salt $r$ and a pair of small polynomials $(s_1, s_2)$ such that 
$s_1 + s_2 h = H(r || m)$.  Furthermore, $s_1$ can be derived fully from $m , r$ and $s_2$ so the signature
is simply given by $(r, s_2)$.  The signature $(s_1,s_2)$ must satisfy $||(s_1,s_2)||^2 \leq \lfloor \beta^2 \rfloor$ where
$\beta = 1.1 \sigma \sqrt{2n}$.  In particular, $\beta^2 = 34034726$ for Level-I parameters and $\beta^2 = 70265242$
for Level-V parameters.  To verify a Falcon signature, one proceeds as described in Algorithm~\ref{algo:falcon_verify}

\begin{algorithm}[!ht] \label{algo:falcon_verify}
\begin{scriptsize}
\caption{\emph{Falcon verification}}
\KwIn{Message $m$, signature $sig = (r,s)$, publick key $pk = h$ and bound $\lfloor \beta^2 \rfloor$}
\KwOut{boolean indicating if signature is valid}
\Begin{
\everypar={\nl}
$c := HashToPoint(r||m,q,n)$ 

$s_2 = Decompress(s, 8 \cdot sbytelen - 328)$

if $(s_2 = \perp)$ then return false

$s_1 = c - s_2 h \bmod q$

if $||(s_1,s_2)||^2 \leq  \lfloor \beta^2 \rfloor$ then return true else return false
}
\end{scriptsize}
\end{algorithm} 


%!TeX root=miden_lattices.tex

\section{Encryption}
\label{section:encryption}

This section specifies a public key encyption scheme over the native finite field $\mathbb{Z}_p$. 

\subsection{Basic Description}

The scheme employs matrix multiplications where the matrix algebra is defined relative to the base ring $R_p = \frac{\mathbb{Z}_p[X]}{(X^n + 1)}$ where $n=64$. The elements of this base ring are polynomials, and so the elements of the matrices and vectors are polynomials. However, the fact that they are polynomials is only relevant to define the multiplication law, which can be computed using the fast NTT-based algorithm described in \S~\ref{section:multiplication-using-ntt}. (The addition law is trivially element-wise addition.) It is natural and fitting to represent these polynomials as vectors of $n=64$ field elements.

Let $\mathbf{G} \in R_p^{m \times m}$ be an arbitrary matrix, and let $\mathbf{a}, \mathbf{b}, \mathbf{c}, \mathbf{d} \in R_p^{m \times 1}$ be vectors of short polynomials. The encryption scheme builds on the following noisy Diffie-Hellman protocol:
\begin{itemize}
\item The square matrix $\mathbf{G}$ is a public parameter known to both Alice and Bob.
\item Alice samples $\mathbf{a}, \mathbf{b}$, computes $\mathbf{A} = \mathbf{G} \mathbf{a} + \mathbf{b}$, and sends $\mathbf{A}$ to Bob.
\item Bob samples $\mathbf{c}, \mathbf{d}$, computes $\mathbf{B} = \mathbf{G}^\mathsf{T} \mathbf{c} + \mathbf{d}$, and sends $\mathbf{B}$ to Alice.
\item Alice receives $\mathbf{B}$ from Bob and computes $K_A = \mathbf{a}^\mathsf{T} \mathbf{B}$.
\item Bob receives $\mathbf{A}$ from Alice and computes $K_B = \mathbf{c}^\mathsf{T} \mathbf{A}$.
\end{itemize}
Alice's view $K_A$ and Bob's view $K_B$ of the shared secret key are close in the following sense. The difference $K_A - K_B = \mathbf{a}^\mathsf{T} \mathbf{d} - \mathbf{b}^\mathsf{T} \mathbf{c}$ is \emph{short} -- every coefficient of this polynomial has a balanced base-$2^{16}$ expansion with a small $\ell_2$-norm. Specifically, the variance of this $\ell_2$-norm is $\sigma^2 \cdot \sqrt{8mn}$, where $\sigma=2$ is the standard deviation in the distribution $\Xi$ of \S~\ref{section:native-parameter-sets}. For reference, for the worst case parameter set this value is roughly 222 whereas for \emph{random} field elements it is roughly $\sqrt{\frac{2^{32}-1}{12}} \approx 18919$. Therefore, with high likelihood Alice and Bob will agree about the top bit of all these 16-bit chunks. 

The encryption scheme embeds the message in the top bit of each 16-bit chunk, and pads the resulting embedding with the noise shared one-time pad. With high likelihood, the noise does not disturb the message.

\subsection{Naive Scheme}

In more detail, the public key encryption scheme consists of the following objects.
\begin{itemize}
 \item The matrix $\mathbf{G} \in R_p^{m \times m}$ is a public parameter.
 \item A secret key is a pair of short vectors $\mathbf{a}, \mathbf{b} \in R_p^{m \times 1}$.
 \item A public key is single vector $\mathbf{A} \in R_p^{m \times 1}$.
 \item A message is a list of 256 bits $m \in \{0,1\}^{256}$.
 \item A ciphertext is a pair $(\mathbf{B}, C) \in R_p^{m \times 1} \times R_p$.
\end{itemize}

We start by defining a functionality that encryption and decryption relies on, namely the embedding and extraction of a message $m \in \{0,1\}^{256}$ into the top bits of 16-bit chunks of a polynomial $f \in R_p$.

\begin{algorithm}[!t] \label{algo:tb4-embed-msg}
\begin{scriptsize}
\caption{\emph{$\mathsf{embed\_msg}$}}
\KwIn{a message $m \in \{0,1\}^{256}$}
\KwOut{a polynomial $M \in R_p$}
\Begin{
\everypar={\nl}
\textbf{return} {$[\sum_{i=0}^3 2^{16i+15} \cdot m[i+4j] : 0 \leq j < 64]$}
}
\end{scriptsize}
\end{algorithm} 
\begin{algorithm}[!t] \label{algo:tb4-extract-msg}
\begin{scriptsize}
\caption{\emph{$\mathsf{extract\_msg}$}}
\KwIn{a polynomial $M \in R_p$}
\KwOut{a message $m \in \{0,1\}^{256}$}
\Begin{
\everypar={\nl}

$m \leftarrow []$

\For{$c \in M$}{
	\For{$i \in \{0,3\}$}{
		$\mathit{chunk} \leftarrow c \mathtt{\&} \mathtt{0xffff}$
		
		$c \leftarrow c \gg 16$
		
		\If{$\mathit{chunk} < 2^{14} \vee 2^{16}-\mathit{chunk} < 2^{14}$}{
			$m \leftarrow m \Vert 0$
		}
		\Else{
			$m \leftarrow m \Vert 1$
		}
	}
}

\textbf{return} {$m$}
}
\end{scriptsize}
\end{algorithm} 

The following algorithms specify the public key encryption scheme. Note that the operations $+$ and $\times$ apply to vectors of 64 field elements. Specifically, these operations compute addition and multiplication in the ring $R_p$. This corresponds to element-wise addition and multiplication of polynomials followed by reduction modulo $X^{64}+1$.

\begin{algorithm}[!t] \label{algo:tb4-keygen-naive}
\begin{scriptsize}
\caption{\emph{$\mathsf{KeyGen}$}}
\KwIn{}
\KwOut{a secret key $\mathit{sk}$ and public key $\mathit{pk}$}
\Begin{
\everypar={\nl}
$\mathbf{a} \leftarrow [[\mathsf{sample\_short\_field\_element}() : 0 \leq i < 64] : 0 \leq j < m]$

$\mathbf{b} \leftarrow [[\mathsf{sample\_short\_field\_element}() : 0 \leq i < 64] : 0 \leq j < m]$

// compute Alice's Diffie-Hellman contribution

$\mathbf{A} \leftarrow [[0 : 0 \leq i < 64] : 0 \leq j < m]$

\For{$i \in \{0, \ldots, m-1\}$}{
	\For{$j \in \{0, \ldots, m-1\}$}{
		$\mathbf{A}[i] \leftarrow \mathbf{A}[i] + \mathbf{G}[i][j] \times \mathbf{a}[j]$
	}
	$\mathbf{A}[i] \leftarrow \mathbf{A}[i] + \mathbf{b}[i]$
}

\textbf{return} {$\mathit{sk} = (\mathbf{a}, \mathbf{b})$, $\mathit{pk} = \mathbf{A}$}
}
\end{scriptsize}
\end{algorithm} 

\begin{algorithm}[!t] \label{algo:tb4-enc-naive}
\begin{scriptsize}
\caption{\emph{$\mathsf{Enc}$}}
\KwIn{a public key $\mathit{pk} = \mathbf{A}$, a message $m \in \{0,1\}^{256}$}
\KwOut{a ciphertext $\mathit{ctxt} = (\mathbf{B}, C)$}
\Begin{
\everypar={\nl}
$\mathbf{c} \leftarrow [[\mathsf{sample\_short\_field\_element}() : 0 \leq i < 64] : 0 \leq j < m]$

$\mathbf{d} \leftarrow [[\mathsf{sample\_short\_field\_element}() : 0 \leq i < 64] : 0 \leq j < m]$

// compute Bob's Diffie-Hellman contribution

$\mathbf{B} \leftarrow [[0 : 0 \leq i < 64] : 0 \leq j < m]$

\For{$i \in \{0, \ldots, m-1\}$}{
	\For{$j \in \{0, \ldots, m-1\}$}{
		$\mathbf{B}[i] \leftarrow \mathbf{B}[i] + \mathbf{G}[j][i] \times \mathbf{c}[j]$
	}
	$\mathbf{B}[i] \leftarrow \mathbf{B}[i] + \mathbf{d}[i]$
}

// compute shared noisy one-time pad

$K \leftarrow [\mathsf{sample\_short\_field\_element}() : 0 \leq i < 64]$

\For{$i \in \{0, \ldots, m-1\}$}{
	$K \leftarrow K + \mathbf{c}[i] \times \mathbf{A}[i]$
}

// pad message

$C \leftarrow K + \mathsf{embed\_msg}(m)$

\textbf{return} {$\mathit{ctxt} = (\mathbf{B}, C)$}
}
\end{scriptsize}
\end{algorithm} 

\begin{algorithm}[!t] \label{algo:tb4-dec-naive}
\begin{scriptsize}
\caption{\emph{$\mathsf{Dec}$}}
\KwIn{a secret key $\mathit{sk} = (\mathbf{a}, \mathbf{b})$, a ciphertext $\mathit{ctxt} = (\mathbf{B}, C)$}
\KwOut{a message $m \in \{0, 1\}^{256}$}
\Begin{
\everypar={\nl}

// compute shared noisy one-time pad

$K \leftarrow [0 : 0 \leq i < 64]$

\For{$i \in \{0, \ldots, m-1\}$}{
	$K \leftarrow K + \mathbf{B}[i] \times \mathbf{a}[i]$
}

// unpad message

$M \leftarrow C - K$

\textbf{return} {$\mathsf{extract\_msg}(M)$}
}
\end{scriptsize}
\end{algorithm} 

\subsection{Optimized Scheme}

The strategy to compute multiplications in the polynomial quotient ring $R_p$ by using NTTs followed by INTTs is redundant because almost all INTT maps at the end of one operation are follwed up with an NTT map preparing for the next operation. It pays to represent the polynomials in the frequency domain. The time domain representation is only necessary when sampling small elements and when decoding the message. This observation gives rise to the equivalent public key encryption scheme specified by Algorithms~\ref{algo:tb4-keygen-optimized}-\ref{algo:tb4-enc-optimized}-\ref{algo:tb4-dec-optimized}. Note that the secret key, public key, and ciphertext are now represented in frequency domain. We use $\circ$ to denote the Hadamard (element-wise) product.

\begin{algorithm}[!t] \label{algo:tb4-keygen-optimized}
\begin{scriptsize}
\caption{\emph{$\mathsf{KeyGen}$ \textit{(optimized)}}}
\KwIn{}
\KwOut{a secret key $\mathit{sk}$ and public key $\mathit{pk}$, both represented in frequency domain}
\Begin{
\everypar={\nl}
$\mathbf{a} \leftarrow [[\mathsf{sample\_short\_field\_element}() : 0 \leq i < 64] : 0 \leq j < m]$

$\mathbf{b} \leftarrow [[\mathsf{sample\_short\_field\_element}() : 0 \leq i < 64] : 0 \leq j < m]$

\For{$0 \leq i < m$}{
	$\mathsf{NTT}_{\it sb}(\mathbf{a}[i])$
	
	$\mathsf{NTT}_{\it sb}(\mathbf{b}[i])$
}

// compute Alice's Diffie-Hellman contribution

$\mathbf{A} \leftarrow [[0 : 0 \leq i < 64] : 0 \leq j < m]$

\For{$i \in \{0, \ldots, m-1\}$}{
	\For{$j \in \{0, \ldots, m-1\}$}{
		$\mathbf{A}[i] \leftarrow \mathbf{A}[i] + \mathbf{G}[i][j] \circ \mathbf{a}[j]$
	}
	$\mathbf{A}[i] \leftarrow \mathbf{A}[i] + \mathbf{b}[i]$
}

\textbf{return} {$\mathit{sk} = (\mathbf{a}, \mathbf{b})$, $\mathit{pk} = \mathbf{A}$}
}
\end{scriptsize}
\end{algorithm} 

\begin{algorithm}[!t] \label{algo:tb4-enc-optimized}
\begin{scriptsize}
\caption{\emph{$\mathsf{Enc}$} \textit{(optimized)}}
\KwIn{a public key $\mathit{pk} = \mathbf{A}$ represented in frequency domain, a message $m \in \{0,1\}^{256}$}
\KwOut{a ciphertext $\mathit{ctxt} = (\mathbf{B}, C)$ represented in frequency domain}
\Begin{
\everypar={\nl}
$\mathbf{c} \leftarrow [[\mathsf{sample\_short\_field\_element}() : 0 \leq i < 64] : 0 \leq j < m]$

$\mathbf{d} \leftarrow [[\mathsf{sample\_short\_field\_element}() : 0 \leq i < 64] : 0 \leq j < m]$


\For{$0 \leq i < m$}{
	$\mathsf{NTT}_{\it sb}(\mathbf{c}[i])$
	
	$\mathsf{NTT}_{\it sb}(\mathbf{d}[i])$
}

// compute Bob's Diffie-Hellman contribution

$\mathbf{B} \leftarrow [[0 : 0 \leq i < 64] : 0 \leq j < m]$

\For{$i \in \{0, \ldots, m-1\}$}{
	\For{$j \in \{0, \ldots, m-1\}$}{
		$\mathbf{B}[i] \leftarrow \mathbf{B}[i] + \mathbf{G}[j][i] \circ \mathbf{c}[j]$
	}
	$\mathbf{B}[i] \leftarrow \mathbf{B}[i] + \mathbf{d}[i]$
}

// compute shared noisy one-time pad

$K \leftarrow [\mathsf{sample\_short\_field\_element}() : 0 \leq i < 64]$

$\mathsf{NTT}_{\it sb}(K)$

\For{$i \in \{0, \ldots, m-1\}$}{
	$K \leftarrow K + \mathbf{c}[i] \circ \mathbf{A}[i]$
}

// embed and pad message

$M \leftarrow \mathsf{embed\_msg}(m)$

$\mathsf{NTT}_{\it sb}(M)$

$C \leftarrow K + M$

\textbf{return} {$\mathit{ctxt} = (\mathbf{B}, C)$}
}
\end{scriptsize}
\end{algorithm} 

\begin{algorithm}[!t] \label{algo:tb4-dec-optimized}
\begin{scriptsize}
\caption{\emph{$\mathsf{Dec}$} \textit{(optimized)}}
\KwIn{a secret key $\mathit{sk} = (\mathbf{a}, \mathbf{b})$, a ciphertext $\mathit{ctxt} = (\mathbf{B}, C)$ all represented in frequency domain}
\KwOut{a message $m \in \{0, 1\}^{256}$}
\Begin{
\everypar={\nl}

// compute shared noisy one-time pad

$K \leftarrow [0 : 0 \leq i < 64]$

\For{$i \in \{0, \ldots, m-1\}$}{
	$K \leftarrow K + \mathbf{B}[i] \circ \mathbf{a}[i]$
}

// unpad, intt, and extract message

$M \leftarrow C - K$

$\mathsf{INTT}_{\sf sb}(M)$

\textbf{return} {$\mathsf{extract\_msg}(M)$}
}
\end{scriptsize}
\end{algorithm}

\subsection{Security, Parameters}
\label{section:security-parameters}

The encryption scheme is $\mathsf{IND-CPA}$-secure under the  assumed hardness of a specific member of the decisional Ideal LWE class of problems as described by Bootland \emph{et al.}~\cite{BootlandCSV21}. This specific member is defined below.

\vspace{0.25cm}

\textbf{Hard Problem.} Let $R_p$ and $\Upsilon$ be defined as in \S~\ref{section:native-parameter-sets}. Let $k, l, m$ be small integers. The IMLWE$^*_{k,l,m}$ is to distinguish the distribution (1) from the distribution (2) given the sample $(\mathbf{A}, \mathbf{B}) \in R_p^{k \times l} \times R_p^{k \times m}$ where
\begin{itemize}
\item[(1)] $\mathbf{A}$ is uniformly random and $\mathbf{B} = \mathbf{A} \mathbf{c} + \mathbf{d}$ for some matrices $\mathbf{c} \sim \Upsilon^{l \times m}$ and $\mathbf{d} \sim \Upsilon^{k \times m}$;
\item[(2)] $\mathbf{A}$ and $\mathbf{B}$ are uniformly random.
\end{itemize}

\vspace{0.25cm}

The security reduction constructs a IMLWE solver from a $\mathsf{IND-CPA}$ adversary $\mathcal{A}$. The reduction proceeds in two steps.

In the first step, $\mathcal{A}$ is used to solve a Diffie-Hellman-like problem, where the task is to distinguish the distribution (1) $(\mathbf{G}, \mathbf{A}, \mathbf{B}, K)$ from (2) $(\mathbf{G}, \mathbf{A}, \mathbf{B}, U)$, where all symbols match with their use in the specification of the encryption scheme and where $U \sim \mathcal{U}(R_p)$. In fact, this step is trivial. Supply the adversary with the public parameter $\mathbf{G}$, the public key $\mathbf{A}$, and the ciphertext $(\mathbf{B}, C)$ were $C$ was constructed according to the last lines of $\mathsf{Enc}$. If $\mathcal{A}$ correctly identifies which message was encrypted, then the fourth element of the tuple is not uniform. Conversely, if $\mathcal{A}$'s can do no better than guess at random then the fourth element must be uniform.

In the second step, a solver for this Diffie-Hellman-like problem is used to build one for IMLWE$^*_{m, 1, 1}$. This step is analogous to \S~5.5 of Frodo~\cite{frodo}.

To estimate the hardness of IMLWE$^*_{m, 1, 1}$ we use the tools of Albrecht~\emph{et al.}~\cite{albrecht-estimator} and Ducas~\emph{et al.}~\cite{ducas-estimator}. While these tools estimate the hardness of standard LWE instances of dimension $N$, we argue that they apply also to IMLWE$^*_{m,1,1}$ with $N = 4 m n$, where the factor 4 arises from the integer packing scheme pressing 4 integers into every field element, and the factor $n$ arises from the ring $R_p$.

The aptitude of these estimators warrants a note of caution. The estimators work for a generic $q$-ary lattice. However, the lattice induced by IMLWE$^*_{m,1,1}$ is not $q$-ary and has abundant structure corresponding the algebra over which multiplication is defined. Nevertheless, inspection of the generating matrix shows that the lattice in question is \emph{very close} to $q$-ary, with $q = 2^{16}$. Moreover, the same estimators are used to estimate the hardness of the \emph{structured} (thus not generic) lattice problems underlying Ring- and Module-based lattice cryptosystems. Neither of these caveats are known to give rise to exploitable attacks or even different attack complexity.

\begin{table}
\centering
\caption{Parameters, security, failure probability}
\label{table:parameters}
\begin{tabular}{c||c|c|c|c}
sec. lvl. & $m$ & Albrecht~\emph{et al.} & Ducas~\emph{et al.} & failure probability \\ \hline
128 & 3 & 148.9 & 135.3 & $< 2^{-995\phantom{\vert}}$ \\
192 & 4 & 211.7 & 192.2 & $\sim 2^{-725}$ \\
256 & 6 & 263.5 & 310.8 & $\sim 2^{-331}$
\end{tabular}
\end{table}

\subsection{Homomorphisms and Failure Probability}

The encryption scheme admits two homomorphic operations. First, addition of ciphertexts corresponds to addition of plaintexts modulo 2. Second, multiplication by \emph{sufficiently short} elements of $R_p$ affects the underlying plaintexts in the same way. Both operations can lead to a decryption failure, even if the operand ciphertexts do not, although this event happens with small probability.

To compute an approximation of the probability of decryption failure after any number of homomorphic operations, it is necessary to represent various distributions on a single packed integer. To make this calculation feasible, it is advisable to restrict distributions to the range $[-2^{14},2^{14}]$ by truncation. The probability of decryption failure corresponds to the distance between 1 and the sum of all probabilities of integers in this range. Starting from the distribution of small integers $\Xi$, this distribution evolves as follows.
\begin{itemize}
\item Multiplication of integers corresponds to convolution of their probability distributions.
\item Multiplication of field elements gives rise to at least one packed integer consisting of the sum of 8 products of original packed integers.
\item Multiplication of polynomials in $R_p$ generates a polynomial where each coefficient consists of the sum of two products of field elements.
\item Multiplication of an $l \times m$ matrix of polynomials by an $m \times 1$ vector of polynomials generates an $l \times 1$ vector where each coordinate consists of the sum of $m$ products of polynomials.
\end{itemize}
It is feasible to apply the same homomorphic circuit to the distribution of small elements. The failure probability of decryption of the output ciphertexts is approximately the distance of the sum of this distribution from 1.



%!TeX root=miden_lattices.tex
\section{Post-quantum commitments}

This section specifies a publicly rerandomizable commitment scheme over the native finite field $\mathbb{Z}_p$. A basic commitment scheme consists of two functions:
\begin{itemize}
 \item $\mathsf{commit}$, takes a message and some randomness and outputs a \emph{commitment} along with some \emph{decommitment information}.
 \item $\mathsf{verify}$, takes a commitment, a message, decommitment information, and outputs a bit indicating whether the commitment is valid.
\end{itemize}

A commitment scheme is \emph{publicly rerandomizable} when third parties can derive a new commitment so that:
\begin{itemize}
\item[a)] The new commitment is unlinkable to the original commitment except by the party that produced the original commitment or the party derived the new one.
\item[b)] The party that produced the original commitment can open the new one as well as the old one, but only to the same message.
\end{itemize}

We build this functionality using the ring $R_p = \frac{\mathbb{Z}_p}{\langle X^n + 1\rangle}$ and associated tools as follows. Let $\mathbf{G} \in R_p^{m \times m}$ be a pseudorandom $m \times m$ matrix consisting of polynomials, and let $\mathbf{a}, \mathbf{b}, \mathbf{c}, \mathbf{d} \in R_p^{m \times 1}$ be vectors of short polynomials, and $e \in R_p$ a single short polynomial. Then $(\mathbf{G}, \mathbf{a}^\mathsf{T} \mathbf{G} + \mathbf{b}^\mathsf{T})$ is one LWE sample, and $(\mathbf{G}\mathbf{c} + \mathbf{d}, \mathbf{a}^\mathsf{T} \mathbf{G} \mathbf{c} + \mathbf{b}^\mathsf{T} \mathbf{c} + e)$ is another. Both LWE samples commit to $\mathbf{a}$, in the sense that it is a short approximate solution, and that such solutions are difficult to find. This observation gives rise to the following scheme:
\begin{itemize}
\item The matrix $\mathbf{G} \in R_p^{m \times m}$ is a public parameter.
\item Cory the committer feeds the message $m \in \{0,1\}^*$ into a pseudorandom number generator and uses it to sample $\mathbf{a}$ and $\mathbf{b}$.
\item Cory computes the commitment as $\mathbf{a}^\mathsf{T} \mathbf{G} + \mathbf{b}^\mathsf{T}$.
\item Rachel the rerandomizer samples $\mathbf{c}, \mathbf{d}, e$ and computes the re-randomized commitment as $(\mathbf{G}\mathbf{c} + \mathbf{d}, \mathbf{a}^\mathsf{T} \mathbf{G} \mathbf{c} + \mathbf{b}^\mathsf{T} \mathbf{c} + e)$.
\item To open a commitment, Cory supplies $m$. From this message, $\mathbf{a}$ can be determined, and it can be verified to be a short approximate solution to the matching LWE sample.
\end{itemize}

\subsection{Naive Scheme}

The scheme consists of four functions, relative to a message space $\mathcal{M}$ and randomness $\mathcal{R}$, the latter of which doubles as the space of decommitment information. The commitment has a different data structure before and after rerandomization: before it is $R_p^{1 \times m}$, whereas after it is $R_p^{m \times 1} \times R_p$.
\begin{itemize}
\item $\mathsf{Commit} : \mathcal{M} \times \mathcal{R} \rightarrow R_p^{1 \times m} \times \mathcal{R}$
\item $\mathsf{VerifyRaw} : \mathcal{M} \times \mathcal{R} \times R_p^{\phantom{\vert}1 \times m} \rightarrow \{\mathsf{True},\mathsf{False}\}$
\item $\mathsf{Rerandomize} : R_p^{1 \times m} \rightarrow (R_p^{\phantom{\vert}m \times 1} \times R_p)$
\item $\mathsf{VerifyRerandomized} : \mathcal{M} \times \mathcal{R} \times (R_p^{\phantom{\vert}m \times 1} \times R_p) \rightarrow \{\mathsf{True},\mathsf{False}\}$
\end{itemize}
In addition to this interface we need a pseudorandom mapping from message-randomness pairs to a short vector of polynomials. We construct this manually, starting from a cryptographically secure extendable output function (XOF) to sample uniform bytes, followed by sampler to send these uniform bytes to short polynomials. We overload the function $\mathsf{sample\_short\_field\_element}$ so that it can use the argument as random bits if it is supplied; otherwise the bits are sampled at random.
\begin{itemize}
\item $\mathsf{xof} : \mathcal{M} \times \mathcal{R} \times \mathbb{N} \rightarrow (\{0, 1\}^8)^*$
\end{itemize}
Lastly, we need a procedure to test whether a given polynomial is short enough. To this end we recycle the $\mathsf{extract\_msg}$ function. The polynomial is short enough if this function returns all zeros.

\begin{algorithm}[!t] \label{algo:tb4-commit-naive}
\begin{scriptsize}
\caption{\emph{$\mathsf{Commit}$}}
\KwIn{a message $t \in \mathcal{M}$ and randomness $r \in \mathcal{R}$}
\KwOut{a commitment $\mathbf{A} \in R_p^{1 \times m}$ and decommitment information}
\Begin{
\everypar={\nl}
$\mathit{uniform\_bytes} \leftarrow \mathsf{xof}(t, r, 256 \cdot m)$

$\mathit{ch} \leftarrow [\mathit{uniform\_bytes}[128 \cdot i : 128 \cdot (i+1)] : i \in \{0, \ldots, 2m-1\}]$

$\mathbf{a} \leftarrow [[\mathsf{sample\_short\_field\_element}(\textit{ch}[128 j + 2i: 128 j + 2(i+1)]) : 0 \leq i < 64] : 0 \leq j < m]$

$\mathbf{b} \leftarrow [[\mathsf{sample\_short\_field\_element}(\textit{ch}[128 j + 2i: 128 j + 2(i+1)]) : 0 \leq i < 64] : m \leq j < 2m]$

$\mathbf{A} \leftarrow [[0 : 0 \leq i < 64] : 0 \leq j < m]$

\For{$i \in \{0, \ldots, m-1\}$}{
	\For{$j \in \{0, \ldots, m-1\}$}{
		$\mathbf{A}[i] \leftarrow \mathbf{A}[i] + \mathbf{G}[i][j] \times \mathbf{a}[j]$
	}
	$\mathbf{A}[i] \leftarrow \mathbf{A}[i] + \mathbf{b}[i]$
}

\textbf{return} {$\mathit{commitment} = \mathbf{A}, \mathit{decommitment\_info} = r$}
}
\end{scriptsize}
\end{algorithm} 

\begin{algorithm}[!t] \label{algo:tb4-verify-raw-naive}
\begin{scriptsize}
\caption{\emph{$\mathsf{Verify}$}}
\KwIn{a message $t \in \mathcal{M}$, decommitment information $r \in \mathcal{R}$, a commitment $\mathit{com} \in R_p^{1 \times m}$}
\KwOut{$\mathsf{True}$ or $\mathsf{False}$}
\Begin{
\everypar={\nl}
$\mathit{uniform\_bytes} \leftarrow \mathsf{xof}(t, r, 256 \cdot m)$

$\mathit{ch} \leftarrow [\mathit{uniform\_bytes}[128 \cdot i : 128 \cdot (i+1)] : i \in \{0, \ldots, 2m-1\}]$

$\mathbf{a} \leftarrow [[\mathsf{sample\_short\_field\_element}(\textit{ch}[128 j + 2i: 128 j + 2(i+1)]) : 0 \leq i < 64] : 0 \leq j < m]$

$\mathbf{b} \leftarrow [[\mathsf{sample\_short\_field\_element}(\textit{ch}[128 j + 2i: 128 j + 2(i+1)]) : 0 \leq i < 64] : m \leq j < 2m]$

$\mathbf{A} \leftarrow [[0 : 0 \leq i < 64] : 0 \leq j < m]$

\For{$i \in \{0, \ldots, m-1\}$}{
	\For{$j \in \{0, \ldots, m-1\}$}{
		$\mathbf{A}[i] \leftarrow \mathbf{A}[i] + \mathbf{G}[i][j] \times \mathbf{a}[j]$
	}
	$\mathbf{A}[i] \leftarrow \mathbf{A}[i] + \mathbf{b}[i]$
}

\textbf{return} {$\mathit{com} \stackrel{?}{=} \mathbf{A}$}
}
\end{scriptsize}
\end{algorithm} 

\begin{algorithm}[!t] \label{algo:tb4-rerandomize-naive}
\begin{scriptsize}
\caption{\emph{$\mathsf{Rerandomize}$}}
\KwIn{a commitment $\mathbf{A} \in R_p^{1 \times m}$}
\KwOut{a rerandomized commitment $(\mathbf{B}, K) \in R_p^{m \times 1} \times R_p$}
\Begin{
\everypar={\nl}
$\mathbf{c} \leftarrow [[\mathsf{sample\_short\_field\_element}() : 0 \leq i < 64] : 0 \leq j < m]$

$\mathbf{d} \leftarrow [[\mathsf{sample\_short\_field\_element}() : 0 \leq i < 64] : m \leq j < 2m]$

$\mathbf{B} \leftarrow [[0 : 0 \leq i < 64] : 0 \leq j < m]$

\For{$i \in \{0, \ldots, m-1\}$}{
	\For{$j \in \{0, \ldots, m-1\}$}{
		$\mathbf{B}[i] \leftarrow \mathbf{B}[i] + \mathbf{G}[j][i] \times \mathbf{c}[j]$
	}
	$\mathbf{B}[i] \leftarrow \mathbf{B}[i] + \mathbf{d}[i]$
}

$e \leftarrow [\mathsf{sample\_short\_field\_element}() : 0 \leq i < 64]$

$K \leftarrow [0: 0 \leq i < 64]$

\For{$i \in \{0, \ldots, m-1\}$}{
	\For{$j \in \{0, \ldots, m-1\}$}{
		$K \leftarrow K + \mathbf{A}[i] \times \mathbf{c}[i]$
	}
	$K \leftarrow K + e$
}

\textbf{return} {$\mathit{recom} = (\mathbf{B}, K)$}
}
\end{scriptsize}
\end{algorithm} 

\begin{algorithm}[!t] \label{algo:tb4-verify-rerandomized-naive}
\begin{scriptsize}
\caption{\emph{$\mathsf{VerifyRerandomized}$}}
\KwIn{a message $t \in \mathcal{M}$, decommitment information $r \in \mathcal{R}$, and a rerandomized commitment $(\mathbf{B}, K) \in R_p^{m \times 1} \times R_p$}
\KwOut{$\mathsf{True}$ or $\mathsf{False}$}
\Begin{
\everypar={\nl}

$\mathit{uniform\_bytes} \leftarrow \mathsf{xof}(t, r, 256 \cdot m)$

$\mathit{ch} \leftarrow [\mathit{uniform\_bytes}[128 \cdot i : 128 \cdot (i+1)] : i \in \{0, \ldots, 2m-1\}]$

$\mathbf{a} \leftarrow [[\mathsf{sample\_short\_field\_element}(\textit{ch}[128 j + 2i: 128 j + 2(i+1)]) : 0 \leq i < 64] : 0 \leq j < m]$


\For{$i \in \{0, \ldots, m-1\}$}{
	$K \leftarrow K - \mathbf{a}[i] \times \mathbf{B}[i]$
}

\textbf{return} {$\mathsf{extract\_msg}(K) \stackrel{?}{=} 0^{256}$}
}
\end{scriptsize}
\end{algorithm}

\subsection{Optimized Scheme}

Like in the case of the encryption scheme, a lot of cycles are wasted going to and from frequency domain just to compute a multiplication. It is better to represent the relevant objects in frequency domain and map them to and from time domain only when needed. Specifically, NTTs are necessary after sampling short elements, and INTTs before testing the lengths of elements. This observation gives rise to the optimized variant of the scheme, whose algorithms follow.


\begin{algorithm}[!t] \label{algo:tb4-commit-optimized}
\begin{scriptsize}
\caption{\emph{$\mathsf{Commit}$ (Optimized)}}
\KwIn{a message $t \in \mathcal{M}$ and randomness $r \in \mathcal{R}$}
\KwOut{a commitment $\mathbf{A} \in R_p^{1 \times m}$ and decommitment information}
\Begin{
\everypar={\nl}
$\mathit{uniform\_bytes} \leftarrow \mathsf{xof}(t, r, 256 \cdot m)$

$\mathit{ch} \leftarrow [\mathit{uniform\_bytes}[128 \cdot i : 128 \cdot (i+1)] : i \in \{0, \ldots, 2m-1\}]$

$\mathbf{a} \leftarrow [[\mathsf{sample\_short\_field\_element}(\textit{ch}[128 j + 2i: 128 j + 2(i+1)]) : 0 \leq i < 64] : 0 \leq j < m]$

\For{$0 \leq i < m$}{
	$\mathsf{NTT}_{\it sb}(\mathbf{a}[i])$
}

$\mathbf{b} \leftarrow [[\mathsf{sample\_short\_field\_element}(\textit{ch}[128 j + 2i: 128 j + 2(i+1)]) : 0 \leq i < 64] : m \leq j < 2m]$

\For{$0 \leq i < m$}{
	$\mathsf{NTT}_{\it sb}(\mathbf{b}[i])$
}


$\mathbf{A} \leftarrow [[0 : 0 \leq i < 64] : 0 \leq j < m]$

\For{$i \in \{0, \ldots, m-1\}$}{
	\For{$j \in \{0, \ldots, m-1\}$}{
		$\mathbf{A}[i] \leftarrow \mathbf{A}[i] + \mathbf{G}[i][j] \circ \mathbf{a}[j]$
	}
	$\mathbf{A}[i] \leftarrow \mathbf{A}[i] + \mathbf{b}[i]$
}

\textbf{return} {$\mathit{commitment} = \mathbf{A}, \mathit{decommitment\_info} = r$}
}
\end{scriptsize}
\end{algorithm} 

\begin{algorithm}[!t] \label{algo:tb4-verify-raw-optimized}
\begin{scriptsize}
\caption{\emph{$\mathsf{VerifyRaw}$ \textit{(Optimized)}}}
\KwIn{a message $t \in \mathcal{M}$, decommitment information $r \in \mathcal{R}$, a commitment $\mathit{com} \in R_p^{1 \times m}$}
\KwOut{$\mathsf{True}$ or $\mathsf{False}$}
\Begin{
\everypar={\nl}
$\mathit{uniform\_bytes} \leftarrow \mathsf{xof}(t, r, 256 \cdot m)$

$\mathit{ch} \leftarrow [\mathit{uniform\_bytes}[128 \cdot i : 128 \cdot (i+1)] : i \in \{0, \ldots, 2m-1\}]$

$\mathbf{a} \leftarrow [[\mathsf{sample\_short\_field\_element}(\textit{ch}[128 j + 2i: 128 j + 2(i+1)]) : 0 \leq i < 64] : 0 \leq j < m]$



$\mathbf{b} \leftarrow [[\mathsf{sample\_short\_field\_element}(\textit{ch}[128 j + 2i: 128 j + 2(i+1)]) : 0 \leq i < 64] : m \leq j < 2m]$

\For{$0 \leq i < m$}{
	$\mathsf{NTT}_{\it sb}(\mathbf{a}[i])$
	
	$\mathsf{NTT}_{\it sb}(\mathbf{b}[i])$
}


$\mathbf{A} \leftarrow [[0 : 0 \leq i < 64] : 0 \leq j < m]$

\For{$i \in \{0, \ldots, m-1\}$}{
	\For{$j \in \{0, \ldots, m-1\}$}{
		$\mathbf{A}[i] \leftarrow \mathbf{A}[i] + \mathbf{G}[i][j] \circ \mathbf{a}[j]$
	}
	$\mathbf{A}[i] \leftarrow \mathbf{A}[i] + \mathbf{b}[i]$
}

\textbf{return} {$\mathit{com} \stackrel{?}{=} \mathbf{A}$}
}
\end{scriptsize}
\end{algorithm} 

\begin{algorithm}[!t] \label{algo:tb4-rerandomize-optimized}
\begin{scriptsize}
\caption{\emph{$\mathsf{Rerandomize}$ \textit{(Optimized)}}}
\KwIn{a commitment $\mathbf{A} \in R_p^{1 \times m}$}
\KwOut{a rerandomized commitment $(\mathbf{B}, K) \in R_p^{m \times 1} \times R_p$}
\Begin{
\everypar={\nl}
$\mathbf{c} \leftarrow [[\mathsf{sample\_short\_field\_element}() : 0 \leq i < 64] : 0 \leq j < m]$

$\mathbf{d} \leftarrow [[\mathsf{sample\_short\_field\_element}() : 0 \leq i < 64] : m \leq j < 2m]$

\For{$0 \leq i < m$}{
	$\mathsf{NTT}_{\it sb}(\mathbf{c}[i])$
	
	$\mathsf{NTT}_{\it sb}(\mathbf{d}[i])$
}


$\mathbf{B} \leftarrow [[0 : 0 \leq i < 64] : 0 \leq j < m]$

\For{$i \in \{0, \ldots, m-1\}$}{
	\For{$j \in \{0, \ldots, m-1\}$}{
		$\mathbf{B}[i] \leftarrow \mathbf{B}[i] + \mathbf{G}[j][i] \circ \mathbf{c}[j]$
	}
	$\mathbf{B}[i] \leftarrow \mathbf{B}[i] + \mathbf{d}[i]$
}

$K \leftarrow [\mathsf{sample\_short\_field\_element}() : 0 \leq i < 64]$ // $d \sim \Xi$

$\mathsf{NTT}_{\it sb}(K)$

\For{$i \in \{0, \ldots, m-1\}$}{
		$K \leftarrow K + \mathbf{A}[i] \circ \mathbf{c}[i]$
}

\textbf{return} {$\mathit{recom} = (\mathbf{B}, K)$}
}
\end{scriptsize}
\end{algorithm} 

\begin{algorithm}[!t] \label{algo:tb4-verify-rerandomized-optimized}
\begin{scriptsize}
\caption{\emph{$\mathsf{VerifyRerandomized}$ \textit{(Optimized)}}}
\KwIn{a message $t \in \mathcal{M}$, decommitment information $r \in \mathcal{R}$, and a rerandomized commitment $(\mathbf{B}, K) \in R_p^{m \times 1} \times R_p$}
\KwOut{$\mathsf{True}$ or $\mathsf{False}$}
\Begin{
\everypar={\nl}

$\mathit{uniform\_bytes} \leftarrow \mathsf{xof}(t, r, 256 \cdot m)$

$\mathit{ch} \leftarrow [\mathit{uniform\_bytes}[128 \cdot i : 128 \cdot (i+1)] : i \in \{0, \ldots, 2m-1\}]$

$\mathbf{a} \leftarrow [[\mathsf{sample\_short\_field\_element}(\textit{ch}[128 j + 2i: 128 j + 2(i+1)]) : 0 \leq i < 64] : 0 \leq j < m]$

\For{$0 \leq i < m$}{
	$\mathsf{NTT}_{\it sb}(\mathbf{a}[i])$
}

\For{$i \in \{0, \ldots, m-1\}$}{
	$K \leftarrow K - \mathbf{a}[i] \circ \mathbf{B}[i]$
}

$\mathsf{INTT}_{\it sb}(K)$

\textbf{return} {$\mathsf{extract\_msg}(K) \stackrel{?}{=} 0^{256}$}
}
\end{scriptsize}
\end{algorithm}

\subsection{Security, Parameters, and Failure Probability}

The security analysis and failure probability analysis reduces to analyses done for the encryption scheme. As a result, we can reuse the same parameters for the same target security levels and achieve the same failure probabilities. The table summarizing this is Table~\ref{table:parameters}. What remains to be said here is why these properties reduce to prior analyses.

\textbf{Correctness.} Correctness of $\mathsf{VerifyRaw}$ follows from construction. Correctness of $\mathsf{VerifyRerandomized}$ is more intricate. This function returns $\mathsf{True}$ if the noise term
\begin{equation*}
\mathbf{b}^\mathsf{T} \mathbf{c} + e  - \mathbf{a}^\mathsf{T} \mathbf{d}
\end{equation*}
is larger than $2^{14}$ in any one packed digit of any coefficient. The probability of this event is (marginally) less than the probability of a decryption failure.

\textbf{Binding.} The binding property decomposes into that of commitments before rerandomization and that after.

Before rerandomization: suppose a commitment $\mathbf{A}$ has two valid openings: $(t_0, r_0)$ and $(t_1, r_1)$. Let $(\mathbf{a_0}, \mathbf{b_0})$ and $(\mathbf{a_1}, \mathbf{b_1})$ be the pairs of short vectors of polynomials that arise after seeding $(t_0, r_0)$ or $(t_1, r_1)$ into the XOF and sampling short vectors of polynomials from the resulting output stream. Distinguish two cases:
\begin{itemize}
 \item $(\mathbf{a_0}, \mathbf{b_0}) \neq (\mathbf{a_1}, \mathbf{b_1})$. Over the random coins of $(\mathbf{a_1}, \mathbf{b_1})$, the probability that $\mathbf{a_0}^\mathsf{T} \mathbf{G} + \mathbf{b_0}^\mathsf{T} = \mathbf{a_1}^\mathsf{T} \mathbf{G} + \mathbf{b_0}^\mathsf{T}$ is approximately $|R_p^m|^{-1} \approx 2^{-4096m}$. Therefore, the probability of sampling distinct short vectors in the same lattice from the XOF is negligible for adversaries with bounded time.
 \item $(\mathbf{a_0}, \mathbf{b_0}) = (\mathbf{a_1}, \mathbf{b_1})$. Each binomial coefficient has about 3.047 bits of entropy. Every field element has 4 binomial coefficient; every polynomial 64 field elements, and every vector $m$ polynomials. The total is roughly $768 m$ bits of entropy. The cost of finding a collision in this distribution is on the order of $2^{384 m}$.
\end{itemize}

\textbf{Hiding.} Ignore the cost of attacking the XOF. The attacker who obtains $(\mathbf{a}, \mathbf{b})$ from $\mathbf{A}$ can be used to undermine the security of the encryption scheme. The attacker who obtains $\mathbf{a}$ from $(\mathbf{B}, K)$ can likewise be used to undermine the security of the encryption scheme. Therefore, the hiding property of the commitment scheme is at least as strong as the IND-CPA of the encryption scheme.

\textbf{Unlinkability.} The adversary who can determine whether a pair $(\mathbf{A}, (\mathbf{B}, K))$ is matching or mismatching (i.e., fix the same short solution $(\mathbf{a}, \mathbf{b})$ or not) can be used to win the decisional Diffie-Hellman game. An analogous reduction to that of \S~5.5 of the Frodo paper~\cite{frodo} reduces this adversary to a solver of IMLWE$^*_{m,1,1}$. The hardness analysis of this problem is provided in Section~\ref{section:security-parameters}.



%%%%%%%%%%%%%%%%%%%%%%%%%%%%%%%%%%%%%%%%%%%%%%%%%%%%%%%%%%%%%%%%%%%%%%%%%%%%%%%%

\bibliographystyle{plain}
\bibliography{bib}

\end{document}

