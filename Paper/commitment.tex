%!TeX root=miden_lattices.tex
\section{Post-quantum commitments}
\label{section:commitments}

This section specifies a publicly rerandomizable commitment scheme over the native finite field $\mathbb{Z}_p$. A basic commitment scheme consists of two functions:
\begin{itemize}
 \item $\mathsf{commit}$, takes a message and some randomness and outputs a \emph{commitment} along with some \emph{decommitment information}.
 \item $\mathsf{verify}$, takes a commitment, a message, decommitment information, and outputs a bit indicating whether the commitment is valid.
\end{itemize}

A commitment scheme is \emph{publicly rerandomizable} when third parties can derive a new commitment so that:
\begin{itemize}
\item[a)] The new commitment is unlinkable to the original commitment except by the party that produced the original commitment or the party that derived the new one.
\item[b)] The party that produced the original commitment can open the new one as well as the old one, but only to the same message.
\end{itemize}

We build this functionality using the ring $R_p = \frac{\mathbb{Z}_p}{\langle X^n + 1\rangle}$ and associated tools as follows. Let $\mathbf{G} \in R_p^{m \times m}$ be a pseudorandom $m \times m$ matrix consisting of polynomials, and let $\mathbf{a}, \mathbf{b}, \mathbf{c}, \mathbf{d} \in R_p^{m \times 1}$ be vectors of short polynomials, and $e \in R_p$ a single short polynomial. Then $(\mathbf{G}, \mathbf{a}^\mathsf{T} \mathbf{G} + \mathbf{b}^\mathsf{T})$ is one MLWE sample, and $(\mathbf{G}\mathbf{c} + \mathbf{d}, \mathbf{a}^\mathsf{T} \mathbf{G} \mathbf{c} + \mathbf{b}^\mathsf{T} \mathbf{c} + e)$ is another. Both MLWE samples commit to $\mathbf{a}$, in the sense that it is a short approximate solution, and that such solutions are difficult to find. This observation gives rise to the following scheme:
\begin{itemize}
\item The matrix $\mathbf{G} \in R_p^{m \times m}$ is a public parameter.
\item Cory the committer feeds the message $m \in \{0,1\}^*$ into a pseudorandom number generator and uses it to sample $\mathbf{a}$ and $\mathbf{b}$.
\item Cory computes the commitment as $\mathbf{a}^\mathsf{T} \mathbf{G} + \mathbf{b}^\mathsf{T}$.
\item Rachel the rerandomizer samples $\mathbf{c}, \mathbf{d}, e$ and computes the re-randomized commitment as $(\mathbf{G}\mathbf{c} + \mathbf{d}, \mathbf{a}^\mathsf{T} \mathbf{G} \mathbf{c} + \mathbf{b}^\mathsf{T} \mathbf{c} + e)$.
\item To open a commitment, Cory supplies $m$. From this message, $\mathbf{a}$ can be determined, and it can be verified to be a short approximate solution to the matching LWE sample.
\end{itemize}

\subsection{Naive Scheme}

The scheme consists of four functions, relative to a message space $\mathcal{M}$ and randomness $\mathcal{R}$, the latter of which doubles as the space of decommitment information. The commitment has a different data structure before and after rerandomization: before it is $R_p^{1 \times m}$, whereas after it is $R_p^{m \times 1} \times R_p$.
\begin{itemize}
\item $\mathsf{Commit} : \mathcal{M} \times \mathcal{R} \rightarrow R_p^{1 \times m} \times \mathcal{R}$
\item $\mathsf{VerifyRaw} : \mathcal{M} \times \mathcal{R} \times R_p^{\phantom{\vert}1 \times m} \rightarrow \{\mathsf{True},\mathsf{False}\}$
\item $\mathsf{Rerandomize} : R_p^{1 \times m} \rightarrow (R_p^{\phantom{\vert}m \times 1} \times R_p)$
\item $\mathsf{VerifyRerandomized} : \mathcal{M} \times \mathcal{R} \times (R_p^{\phantom{\vert}m \times 1} \times R_p) \rightarrow \{\mathsf{True},\mathsf{False}\}$
\end{itemize}
In addition to this interface we need a pseudorandom mapping from message-randomness pairs to a short vector of polynomials. We construct this manually, starting from a cryptographically secure extendable output function (XOF) to sample uniform bytes, followed by a sampler to send these uniform bytes to short polynomials. We overload the function $\mathsf{sample\_short\_field\_element}$ so that it can use the argument as random bits if it is supplied; otherwise the bits are sampled at random.
\begin{itemize}
\item $\mathsf{xof} : \mathcal{M} \times \mathcal{R} \times \mathbb{N} \rightarrow (\{0, 1\}^8)^*$
\end{itemize}
Lastly, we need a procedure to test whether a given polynomial is short enough. To this end we recycle the $\mathsf{extract\_msg}$ function. The polynomial is short enough if this function returns all zeros.

\begin{algorithm}[!t] \label{algo:tb4-commit-naive}
\begin{scriptsize}
\caption{\emph{$\mathsf{Commit}$}}
\KwIn{a message $t \in \mathcal{M}$ and randomness $r \in \mathcal{R}$}
\KwOut{a commitment $\mathbf{A} \in R_p^{1 \times m}$ and decommitment information}
\Begin{
\everypar={\nl}
$\mathit{uniform\_bytes} \leftarrow \mathsf{xof}(t, r, 256 \cdot m)$

$\mathit{ch} \leftarrow [\mathit{uniform\_bytes}[128 \cdot i : 128 \cdot (i+1)] : i \in \{0, \ldots, 2m-1\}]$

$\mathbf{a} \leftarrow [[\mathsf{sample\_short\_field\_element}(\textit{ch}[128 j + 2i: 128 j + 2(i+1)]) : 0 \leq i < 64] : 0 \leq j < m]$

$\mathbf{b} \leftarrow [[\mathsf{sample\_short\_field\_element}(\textit{ch}[128 j + 2i: 128 j + 2(i+1)]) : 0 \leq i < 64] : m \leq j < 2m]$

$\mathbf{A} \leftarrow [[0 : 0 \leq i < 64] : 0 \leq j < m]$

\For{$i \in \{0, \ldots, m-1\}$}{
	\For{$j \in \{0, \ldots, m-1\}$}{
		$\mathbf{A}[i] \leftarrow \mathbf{A}[i] + \mathbf{G}[i][j] \times \mathbf{a}[j]$
	}
	$\mathbf{A}[i] \leftarrow \mathbf{A}[i] + \mathbf{b}[i]$
}

\textbf{return} {$\mathit{commitment} = \mathbf{A}, \mathit{decommitment\_info} = r$}
}
\end{scriptsize}
\end{algorithm} 

\begin{algorithm}[!t] \label{algo:tb4-verify-raw-naive}
\begin{scriptsize}
\caption{\emph{$\mathsf{Verify}$}}
\KwIn{a message $t \in \mathcal{M}$, decommitment information $r \in \mathcal{R}$, a commitment $\mathit{com} \in R_p^{1 \times m}$}
\KwOut{$\mathsf{True}$ or $\mathsf{False}$}
\Begin{
\everypar={\nl}
$\mathit{uniform\_bytes} \leftarrow \mathsf{xof}(t, r, 256 \cdot m)$

$\mathit{ch} \leftarrow [\mathit{uniform\_bytes}[128 \cdot i : 128 \cdot (i+1)] : i \in \{0, \ldots, 2m-1\}]$

$\mathbf{a} \leftarrow [[\mathsf{sample\_short\_field\_element}(\textit{ch}[128 j + 2i: 128 j + 2(i+1)]) : 0 \leq i < 64] : 0 \leq j < m]$

$\mathbf{b} \leftarrow [[\mathsf{sample\_short\_field\_element}(\textit{ch}[128 j + 2i: 128 j + 2(i+1)]) : 0 \leq i < 64] : m \leq j < 2m]$

$\mathbf{A} \leftarrow [[0 : 0 \leq i < 64] : 0 \leq j < m]$

\For{$i \in \{0, \ldots, m-1\}$}{
	\For{$j \in \{0, \ldots, m-1\}$}{
		$\mathbf{A}[i] \leftarrow \mathbf{A}[i] + \mathbf{G}[i][j] \times \mathbf{a}[j]$
	}
	$\mathbf{A}[i] \leftarrow \mathbf{A}[i] + \mathbf{b}[i]$
}

\textbf{return} {$\mathit{com} \stackrel{?}{=} \mathbf{A}$}
}
\end{scriptsize}
\end{algorithm} 

\begin{algorithm}[!t] \label{algo:tb4-rerandomize-naive}
\begin{scriptsize}
\caption{\emph{$\mathsf{Rerandomize}$}}
\KwIn{a commitment $\mathbf{A} \in R_p^{1 \times m}$}
\KwOut{a rerandomized commitment $(\mathbf{B}, K) \in R_p^{m \times 1} \times R_p$}
\Begin{
\everypar={\nl}
$\mathbf{c} \leftarrow [[\mathsf{sample\_short\_field\_element}() : 0 \leq i < 64] : 0 \leq j < m]$

$\mathbf{d} \leftarrow [[\mathsf{sample\_short\_field\_element}() : 0 \leq i < 64] : m \leq j < 2m]$

$\mathbf{B} \leftarrow [[0 : 0 \leq i < 64] : 0 \leq j < m]$

\For{$i \in \{0, \ldots, m-1\}$}{
	\For{$j \in \{0, \ldots, m-1\}$}{
		$\mathbf{B}[i] \leftarrow \mathbf{B}[i] + \mathbf{G}[j][i] \times \mathbf{c}[j]$
	}
	$\mathbf{B}[i] \leftarrow \mathbf{B}[i] + \mathbf{d}[i]$
}

$e \leftarrow [\mathsf{sample\_short\_field\_element}() : 0 \leq i < 64]$

$K \leftarrow [0: 0 \leq i < 64]$

\For{$i \in \{0, \ldots, m-1\}$}{
	\For{$j \in \{0, \ldots, m-1\}$}{
		$K \leftarrow K + \mathbf{A}[i] \times \mathbf{c}[i]$
	}
	$K \leftarrow K + e$
}

\textbf{return} {$\mathit{recom} = (\mathbf{B}, K)$}
}
\end{scriptsize}
\end{algorithm} 

\begin{algorithm}[!t] \label{algo:tb4-verify-rerandomized-naive}
\begin{scriptsize}
\caption{\emph{$\mathsf{VerifyRerandomized}$}}
\KwIn{a message $t \in \mathcal{M}$, decommitment information $r \in \mathcal{R}$, and a rerandomized commitment $(\mathbf{B}, K) \in R_p^{m \times 1} \times R_p$}
\KwOut{$\mathsf{True}$ or $\mathsf{False}$}
\Begin{
\everypar={\nl}

$\mathit{uniform\_bytes} \leftarrow \mathsf{xof}(t, r, 256 \cdot m)$

$\mathit{ch} \leftarrow [\mathit{uniform\_bytes}[128 \cdot i : 128 \cdot (i+1)] : i \in \{0, \ldots, 2m-1\}]$

$\mathbf{a} \leftarrow [[\mathsf{sample\_short\_field\_element}(\textit{ch}[128 j + 2i: 128 j + 2(i+1)]) : 0 \leq i < 64] : 0 \leq j < m]$


\For{$i \in \{0, \ldots, m-1\}$}{
	$K \leftarrow K - \mathbf{a}[i] \times \mathbf{B}[i]$
}

\textbf{return} {$\mathsf{extract\_msg}(K) \stackrel{?}{=} 0^{256}$}
}
\end{scriptsize}
\end{algorithm}

\subsection{Optimized Scheme}

Like in the case of the encryption scheme, a lot of cycles are wasted going to and from frequency domain just to compute a multiplication. It is better to represent the relevant objects in frequency domain and map them to and from time domain only when needed. Specifically, NTTs are necessary after sampling short elements, and INTTs before testing the lengths of elements. This observation gives rise to the optimized variant of the scheme, whose algorithms follow.


\begin{algorithm}[!t] \label{algo:tb4-commit-optimized}
\begin{scriptsize}
\caption{\emph{$\mathsf{Commit}$ (Optimized)}}
\KwIn{a message $t \in \mathcal{M}$ and randomness $r \in \mathcal{R}$}
\KwOut{a commitment $\mathbf{A} \in R_p^{1 \times m}$ and decommitment information}
\Begin{
\everypar={\nl}
$\mathit{uniform\_bytes} \leftarrow \mathsf{xof}(t, r, 256 \cdot m)$

$\mathit{ch} \leftarrow [\mathit{uniform\_bytes}[128 \cdot i : 128 \cdot (i+1)] : i \in \{0, \ldots, 2m-1\}]$

$\mathbf{a} \leftarrow [[\mathsf{sample\_short\_field\_element}(\textit{ch}[128 j + 2i: 128 j + 2(i+1)]) : 0 \leq i < 64] : 0 \leq j < m]$

\For{$0 \leq i < m$}{
	$\mathsf{NTT}_{\it sb}(\mathbf{a}[i])$
}

$\mathbf{b} \leftarrow [[\mathsf{sample\_short\_field\_element}(\textit{ch}[128 j + 2i: 128 j + 2(i+1)]) : 0 \leq i < 64] : m \leq j < 2m]$

\For{$0 \leq i < m$}{
	$\mathsf{NTT}_{\it sb}(\mathbf{b}[i])$
}


$\mathbf{A} \leftarrow [[0 : 0 \leq i < 64] : 0 \leq j < m]$

\For{$i \in \{0, \ldots, m-1\}$}{
	\For{$j \in \{0, \ldots, m-1\}$}{
		$\mathbf{A}[i] \leftarrow \mathbf{A}[i] + \mathbf{G}[i][j] \circ \mathbf{a}[j]$
	}
	$\mathbf{A}[i] \leftarrow \mathbf{A}[i] + \mathbf{b}[i]$
}

\textbf{return} {$\mathit{commitment} = \mathbf{A}, \mathit{decommitment\_info} = r$}
}
\end{scriptsize}
\end{algorithm} 

\begin{algorithm}[!t] \label{algo:tb4-verify-raw-optimized}
\begin{scriptsize}
\caption{\emph{$\mathsf{VerifyRaw}$ \textit{(Optimized)}}}
\KwIn{a message $t \in \mathcal{M}$, decommitment information $r \in \mathcal{R}$, a commitment $\mathit{com} \in R_p^{1 \times m}$}
\KwOut{$\mathsf{True}$ or $\mathsf{False}$}
\Begin{
\everypar={\nl}
$\mathit{uniform\_bytes} \leftarrow \mathsf{xof}(t, r, 256 \cdot m)$

$\mathit{ch} \leftarrow [\mathit{uniform\_bytes}[128 \cdot i : 128 \cdot (i+1)] : i \in \{0, \ldots, 2m-1\}]$

$\mathbf{a} \leftarrow [[\mathsf{sample\_short\_field\_element}(\textit{ch}[128 j + 2i: 128 j + 2(i+1)]) : 0 \leq i < 64] : 0 \leq j < m]$



$\mathbf{b} \leftarrow [[\mathsf{sample\_short\_field\_element}(\textit{ch}[128 j + 2i: 128 j + 2(i+1)]) : 0 \leq i < 64] : m \leq j < 2m]$

\For{$0 \leq i < m$}{
	$\mathsf{NTT}_{\it sb}(\mathbf{a}[i])$
	
	$\mathsf{NTT}_{\it sb}(\mathbf{b}[i])$
}


$\mathbf{A} \leftarrow [[0 : 0 \leq i < 64] : 0 \leq j < m]$

\For{$i \in \{0, \ldots, m-1\}$}{
	\For{$j \in \{0, \ldots, m-1\}$}{
		$\mathbf{A}[i] \leftarrow \mathbf{A}[i] + \mathbf{G}[i][j] \circ \mathbf{a}[j]$
	}
	$\mathbf{A}[i] \leftarrow \mathbf{A}[i] + \mathbf{b}[i]$
}

\textbf{return} {$\mathit{com} \stackrel{?}{=} \mathbf{A}$}
}
\end{scriptsize}
\end{algorithm} 

\begin{algorithm}[!t] \label{algo:tb4-rerandomize-optimized}
\begin{scriptsize}
\caption{\emph{$\mathsf{Rerandomize}$ \textit{(Optimized)}}}
\KwIn{a commitment $\mathbf{A} \in R_p^{1 \times m}$}
\KwOut{a rerandomized commitment $(\mathbf{B}, K) \in R_p^{m \times 1} \times R_p$}
\Begin{
\everypar={\nl}
$\mathbf{c} \leftarrow [[\mathsf{sample\_short\_field\_element}() : 0 \leq i < 64] : 0 \leq j < m]$

$\mathbf{d} \leftarrow [[\mathsf{sample\_short\_field\_element}() : 0 \leq i < 64] : m \leq j < 2m]$

\For{$0 \leq i < m$}{
	$\mathsf{NTT}_{\it sb}(\mathbf{c}[i])$
	
	$\mathsf{NTT}_{\it sb}(\mathbf{d}[i])$
}


$\mathbf{B} \leftarrow [[0 : 0 \leq i < 64] : 0 \leq j < m]$

\For{$i \in \{0, \ldots, m-1\}$}{
	\For{$j \in \{0, \ldots, m-1\}$}{
		$\mathbf{B}[i] \leftarrow \mathbf{B}[i] + \mathbf{G}[j][i] \circ \mathbf{c}[j]$
	}
	$\mathbf{B}[i] \leftarrow \mathbf{B}[i] + \mathbf{d}[i]$
}

$K \leftarrow [\mathsf{sample\_short\_field\_element}() : 0 \leq i < 64]$ // initialize with $e \sim \Upsilon^{64}$

$\mathsf{NTT}_{\it sb}(K)$

\For{$i \in \{0, \ldots, m-1\}$}{
		$K \leftarrow K + \mathbf{A}[i] \circ \mathbf{c}[i]$
}

\textbf{return} {$\mathit{recom} = (\mathbf{B}, K)$}
}
\end{scriptsize}
\end{algorithm} 

\begin{algorithm}[!t] \label{algo:tb4-verify-rerandomized-optimized}
\begin{scriptsize}
\caption{\emph{$\mathsf{VerifyRerandomized}$ \textit{(Optimized)}}}
\KwIn{a message $t \in \mathcal{M}$, decommitment information $r \in \mathcal{R}$, and a rerandomized commitment $(\mathbf{B}, K) \in R_p^{m \times 1} \times R_p$}
\KwOut{$\mathsf{True}$ or $\mathsf{False}$}
\Begin{
\everypar={\nl}

$\mathit{uniform\_bytes} \leftarrow \mathsf{xof}(t, r, 256 \cdot m)$

$\mathit{ch} \leftarrow [\mathit{uniform\_bytes}[128 \cdot i : 128 \cdot (i+1)] : i \in \{0, \ldots, 2m-1\}]$

$\mathbf{a} \leftarrow [[\mathsf{sample\_short\_field\_element}(\textit{ch}[128 j + 2i: 128 j + 2(i+1)]) : 0 \leq i < 64] : 0 \leq j < m]$

\For{$0 \leq i < m$}{
	$\mathsf{NTT}_{\it sb}(\mathbf{a}[i])$
}

\For{$i \in \{0, \ldots, m-1\}$}{
	$K \leftarrow K - \mathbf{a}[i] \circ \mathbf{B}[i]$
}

$\mathsf{INTT}_{\it sb}(K)$

\textbf{return} {$\mathsf{extract\_msg}(K) \stackrel{?}{=} 0^{256}$}
}
\end{scriptsize}
\end{algorithm}

\subsection{Security, Parameters, and Failure Probability}

The security analysis and failure probability analysis reduces to analyses done for the encryption scheme. As a result, we can reuse the same parameters for the same target security levels and achieve the same failure probabilities. The table summarizing this is Table~\ref{table:parameters}. What remains to be said here is why these properties reduce to prior analyses.

\textbf{Correctness.} Correctness of $\mathsf{VerifyRaw}$ follows from construction. Correctness of $\mathsf{VerifyRerandomized}$ is more intricate. This function returns $\mathsf{False}$ if the noise term
\begin{equation*}
\mathbf{b}^\mathsf{T} \mathbf{c} + e  - \mathbf{a}^\mathsf{T} \mathbf{d}
\end{equation*}
is larger than $2^{14}$ in any one packed digit of any coefficient. The probability of this event is (marginally) less than the probability of a decryption failure.

\textbf{Binding.} The binding property decomposes into that of commitments before rerandomization and that after.

Before rerandomization: suppose a commitment $\mathbf{A}$ has two valid openings: $(t_0, r_0)$ and $(t_1, r_1)$. Let $(\mathbf{a_0}, \mathbf{b_0})$ and $(\mathbf{a_1}, \mathbf{b_1})$ be the pairs of short vectors of polynomials that arise after seeding $(t_0, r_0)$ or $(t_1, r_1)$ into the XOF and sampling short vectors of polynomials from the resulting output stream. Distinguish two cases:
\begin{itemize}
 \item $(\mathbf{a_0}, \mathbf{b_0}) \neq (\mathbf{a_1}, \mathbf{b_1})$. Over the random coins of $(\mathbf{a_1}, \mathbf{b_1})$, the probability that $\mathbf{a_0}^\mathsf{T} \mathbf{G} + \mathbf{b_0}^\mathsf{T} = \mathbf{a_1}^\mathsf{T} \mathbf{G} + \mathbf{b_0}^\mathsf{T}$ is approximately $|R_p^m|^{-1} \approx 2^{-4096m}$. Therefore, the probability of sampling distinct short vectors in the same lattice from the XOF is negligible for adversaries with bounded time.
 \item $(\mathbf{a_0}, \mathbf{b_0}) = (\mathbf{a_1}, \mathbf{b_1})$. Each binomial coefficient has about 3.047 bits of entropy. Every field element has 4 binomial coefficients; every polynomial 64 field elements, and every vector $m$ polynomials. The total is roughly $768 m$ bits of entropy. The cost of finding a collision in this distribution is on the order of $2^{384 m}$.
\end{itemize}

\textbf{Hiding.} Ignore the cost of attacking the XOF. The attacker who obtains $(\mathbf{a}, \mathbf{b})$ from $\mathbf{A}$ can be used to undermine the security of the encryption scheme. The attacker who obtains $\mathbf{a}$ from $(\mathbf{B}, K)$ can likewise be used to undermine the security of the encryption scheme. Therefore, the hiding property of the commitment scheme is at least as strong as the IND-CPA of the encryption scheme.

\textbf{Unlinkability.} The adversary who can determine whether a pair $(\mathbf{A}, (\mathbf{B}, K))$ is matching or mismatching (i.e., fix the same short solution $(\mathbf{a}, \mathbf{b})$ or not) can be used to win the decisional Diffie-Hellman game. An analogous reduction to that of \S~5.5 of the Frodo paper~\cite{frodo} reduces this adversary to a solver of IMLWE$^*_{m,1,1}$. The hardness analysis of this problem is provided in Section~\ref{section:security-parameters}.
