%!TeX root=miden_lattices.tex

\section{Encryption}
\label{section:encryption}

This section specifies a public key encyption scheme over the native finite field $\mathbb{Z}_p$. 

\subsection{Basic Description}

The scheme employs matrix multiplications where the matrix algebra is defined relative to the base ring $R_p = \frac{\mathbb{Z}_p[X]}{(X^n + 1)}$ where $n=64$. The elements of this base ring are polynomials, and so the elements of the matrices and vectors are polynomials. However, the fact that they are polynomials is only relevant to define the multiplication law, which can be computed using the fast NTT-based algorithm described in \S~\ref{section:multiplication-using-ntt}. (The addition law is trivially element-wise addition.) It is natural and fitting to represent these polynomials as vectors of $n=64$ field elements.

Let $\mathbf{G} \in R_p^{m \times m}$ be an arbitrary matrix, and let $\mathbf{a}, \mathbf{b}, \mathbf{c}, \mathbf{d} \in R_p^{m \times 1}$ be vectors of short polynomials. The encryption scheme builds on the following noisy Diffie-Hellman protocol:
\begin{itemize}
\item The square matrix $\mathbf{G}$ is a public parameter known to both Alice and Bob.
\item Alice samples $\mathbf{a}, \mathbf{b}$, computes $\mathbf{A} = \mathbf{G} \mathbf{a} + \mathbf{b}$, and sends $\mathbf{A}$ to Bob.
\item Bob samples $\mathbf{c}, \mathbf{d}$, computes $\mathbf{B} = \mathbf{G}^\mathsf{T} \mathbf{c} + \mathbf{d}$, and sends $\mathbf{B}$ to Alice.
\item Alice receives $\mathbf{B}$ from Bob and computes $K_A = \mathbf{a}^\mathsf{T} \mathbf{B}$.
\item Bob receives $\mathbf{A}$ from Alice and computes $K_B = \mathbf{c}^\mathsf{T} \mathbf{A}$.
\end{itemize}
Alice's view $K_A$ and Bob's view $K_B$ of the shared secret key are close in the following sense. The difference $K_A - K_B = \mathbf{a}^\mathsf{T} \mathbf{d} - \mathbf{b}^\mathsf{T} \mathbf{c}$ is \emph{short} -- every coefficient of this polynomial has a balanced base-$2^{16}$ expansion with a small $\ell_2$-norm. Specifically, the variance of this $\ell_2$-norm is $\sigma^2 \cdot \sqrt{8mn}$, where $\sigma=2$ is the standard deviation in the distribution $\Xi$ of \S~\ref{section:native-parameter-sets}. For reference, for the worst case parameter set this value is roughly 222 whereas for \emph{random} field elements it is roughly $\sqrt{\frac{2^{32}-1}{12}} \approx 18919$. Therefore, with high likelihood Alice and Bob will agree about the top bit of all these 16-bit chunks. 

The encryption scheme embeds the message in the top bit of each 16-bit chunk, and pads the resulting embedding with the noise shared one-time pad. With high likelihood, the noise does not disturb the message.

\subsection{Naive Scheme}

In more detail, the public key encryption scheme consists of the following objects.
\begin{itemize}
 \item The matrix $\mathbf{G} \in R_p^{m \times m}$ is a public parameter.
 \item A secret key is a pair of short vectors $\mathbf{a}, \mathbf{b} \in R_p^{m \times 1}$.
 \item A public key is single vector $\mathbf{A} \in R_p^{m \times 1}$.
 \item A message is a list of 256 bits $m \in \{0,1\}^{256}$.
 \item A ciphertext is a pair $(\mathbf{B}, C) \in R_p^{m \times 1} \times R_p$.
\end{itemize}

We start by defining a functionality that encryption and decryption relies on, namely the embedding and extraction of a message $m \in \{0,1\}^{256}$ into the top bits of 16-bit chunks of a polynomial $f \in R_p$.

\begin{algorithm}[!t] \label{algo:tb4-embed-msg}
\begin{scriptsize}
\caption{\emph{$\mathsf{embed\_msg}$}}
\KwIn{a message $m \in \{0,1\}^{256}$}
\KwOut{a polynomial $M \in R_p$}
\Begin{
\everypar={\nl}
\textbf{return} {$[\sum_{i=0}^3 2^{16i+15} \cdot m[i+4j] : 0 \leq j < 64]$}
}
\end{scriptsize}
\end{algorithm} 
\begin{algorithm}[!t] \label{algo:tb4-extract-msg}
\begin{scriptsize}
\caption{\emph{$\mathsf{extract\_msg}$}}
\KwIn{a polynomial $M \in R_p$}
\KwOut{a message $m \in \{0,1\}^{256}$}
\Begin{
\everypar={\nl}

$m \leftarrow []$

\For{$c \in M$}{
	\For{$i \in \{0,3\}$}{
		$\mathit{chunk} \leftarrow c \mathtt{\&} \mathtt{0xffff}$
		
		$c \leftarrow c \gg 16$
		
		\If{$\mathit{chunk} < 2^{14} \vee 2^{16}-\mathit{chunk} < 2^{14}$}{
			$m \leftarrow m \Vert 0$
		}
		\Else{
			$m \leftarrow m \Vert 1$
		}
	}
}

\textbf{return} {$m$}
}
\end{scriptsize}
\end{algorithm} 

The following algorithms specify the public key encryption scheme. Note that the operations $+$ and $\times$ apply to vectors of 64 field elements. Specifically, these operations compute addition and multiplication in the ring $R_p$. This corresponds to element-wise addition and multiplication of polynomials followed by reduction modulo $X^{64}+1$.

\begin{algorithm}[!t] \label{algo:tb4-keygen-naive}
\begin{scriptsize}
\caption{\emph{$\mathsf{KeyGen}$}}
\KwIn{}
\KwOut{a secret key $\mathit{sk}$ and public key $\mathit{pk}$}
\Begin{
\everypar={\nl}
$\mathbf{a} \leftarrow [[\mathsf{sample\_short\_field\_element}() : 0 \leq i < 64] : 0 \leq j < m]$

$\mathbf{b} \leftarrow [[\mathsf{sample\_short\_field\_element}() : 0 \leq i < 64] : 0 \leq j < m]$

// compute Alice's Diffie-Hellman contribution

$\mathbf{A} \leftarrow [[0 : 0 \leq i < 64] : 0 \leq j < m]$

\For{$i \in \{0, \ldots, m-1\}$}{
	\For{$j \in \{0, \ldots, m-1\}$}{
		$\mathbf{A}[i] \leftarrow \mathbf{A}[i] + \mathbf{G}[i][j] \times \mathbf{a}[j]$
	}
	$\mathbf{A}[i] \leftarrow \mathbf{A}[i] + \mathbf{b}[i]$
}

\textbf{return} {$\mathit{sk} = (\mathbf{a}, \mathbf{b})$, $\mathit{pk} = \mathbf{A}$}
}
\end{scriptsize}
\end{algorithm} 

\begin{algorithm}[!t] \label{algo:tb4-enc-naive}
\begin{scriptsize}
\caption{\emph{$\mathsf{Enc}$}}
\KwIn{a public key $\mathit{pk} = \mathbf{A}$, a message $m \in \{0,1\}^{256}$}
\KwOut{a ciphertext $\mathit{ctxt} = (\mathbf{B}, C)$}
\Begin{
\everypar={\nl}
$\mathbf{c} \leftarrow [[\mathsf{sample\_short\_field\_element}() : 0 \leq i < 64] : 0 \leq j < m]$

$\mathbf{d} \leftarrow [[\mathsf{sample\_short\_field\_element}() : 0 \leq i < 64] : 0 \leq j < m]$

// compute Bob's Diffie-Hellman contribution

$\mathbf{B} \leftarrow [[0 : 0 \leq i < 64] : 0 \leq j < m]$

\For{$i \in \{0, \ldots, m-1\}$}{
	\For{$j \in \{0, \ldots, m-1\}$}{
		$\mathbf{B}[i] \leftarrow \mathbf{B}[i] + \mathbf{G}[j][i] \times \mathbf{c}[j]$
	}
	$\mathbf{B}[i] \leftarrow \mathbf{B}[i] + \mathbf{d}[i]$
}

// compute shared noisy one-time pad

$K \leftarrow [\mathsf{sample\_short\_field\_element}() : 0 \leq i < 64]$

\For{$i \in \{0, \ldots, m-1\}$}{
	$K \leftarrow K + \mathbf{c}[i] \times \mathbf{A}[i]$
}

// pad message

$C \leftarrow K + \mathsf{embed\_msg}(m)$

\textbf{return} {$\mathit{ctxt} = (\mathbf{B}, C)$}
}
\end{scriptsize}
\end{algorithm} 

\begin{algorithm}[!t] \label{algo:tb4-dec-naive}
\begin{scriptsize}
\caption{\emph{$\mathsf{Dec}$}}
\KwIn{a secret key $\mathit{sk} = (\mathbf{a}, \mathbf{b})$, a ciphertext $\mathit{ctxt} = (\mathbf{B}, C)$}
\KwOut{a message $m \in \{0, 1\}^{256}$}
\Begin{
\everypar={\nl}

// compute shared noisy one-time pad

$K \leftarrow [0 : 0 \leq i < 64]$

\For{$i \in \{0, \ldots, m-1\}$}{
	$K \leftarrow K + \mathbf{B}[i] \times \mathbf{a}[i]$
}

// unpad message

$M \leftarrow C - K$

\textbf{return} {$\mathsf{extract\_msg}(M)$}
}
\end{scriptsize}
\end{algorithm} 

\subsection{Optimized Scheme}

The strategy to compute multiplications in the polynomial quotient ring $R_p$ by using NTTs followed by INTTs is redundant because almost all INTT maps at the end of one operation are follwed up with an NTT map preparing for the next operation. It pays to represent the polynomials in the frequency domain. The time domain representation is only necessary when sampling small elements and when decoding the message. This observation gives rise to the equivalent public key encryption scheme specified by Algorithms~\ref{algo:tb4-keygen-optimized}-\ref{algo:tb4-enc-optimized}-\ref{algo:tb4-dec-optimized}. Note that the secret key, public key, and ciphertext are now represented in frequency domain. We use $\circ$ to denote the Hadamard (element-wise) product.

\begin{algorithm}[!t] \label{algo:tb4-keygen-optimized}
\begin{scriptsize}
\caption{\emph{$\mathsf{KeyGen}$ \textit{(optimized)}}}
\KwIn{}
\KwOut{a secret key $\mathit{sk}$ and public key $\mathit{pk}$, both represented in frequency domain}
\Begin{
\everypar={\nl}
$\mathbf{a} \leftarrow [[\mathsf{sample\_short\_field\_element}() : 0 \leq i < 64] : 0 \leq j < m]$

$\mathbf{b} \leftarrow [[\mathsf{sample\_short\_field\_element}() : 0 \leq i < 64] : 0 \leq j < m]$

\For{$0 \leq i < m$}{
	$\mathsf{NTT}_{\it sb}(\mathbf{a}[i])$
	
	$\mathsf{NTT}_{\it sb}(\mathbf{b}[i])$
}

// compute Alice's Diffie-Hellman contribution

$\mathbf{A} \leftarrow [[0 : 0 \leq i < 64] : 0 \leq j < m]$

\For{$i \in \{0, \ldots, m-1\}$}{
	\For{$j \in \{0, \ldots, m-1\}$}{
		$\mathbf{A}[i] \leftarrow \mathbf{A}[i] + \mathbf{G}[i][j] \circ \mathbf{a}[j]$
	}
	$\mathbf{A}[i] \leftarrow \mathbf{A}[i] + \mathbf{b}[i]$
}

\textbf{return} {$\mathit{sk} = (\mathbf{a}, \mathbf{b})$, $\mathit{pk} = \mathbf{A}$}
}
\end{scriptsize}
\end{algorithm} 

\begin{algorithm}[!t] \label{algo:tb4-enc-optimized}
\begin{scriptsize}
\caption{\emph{$\mathsf{Enc}$} \textit{(optimized)}}
\KwIn{a public key $\mathit{pk} = \mathbf{A}$ represented in frequency domain, a message $m \in \{0,1\}^{256}$}
\KwOut{a ciphertext $\mathit{ctxt} = (\mathbf{B}, C)$ represented in frequency domain}
\Begin{
\everypar={\nl}
$\mathbf{c} \leftarrow [[\mathsf{sample\_short\_field\_element}() : 0 \leq i < 64] : 0 \leq j < m]$

$\mathbf{d} \leftarrow [[\mathsf{sample\_short\_field\_element}() : 0 \leq i < 64] : 0 \leq j < m]$


\For{$0 \leq i < m$}{
	$\mathsf{NTT}_{\it sb}(\mathbf{c}[i])$
	
	$\mathsf{NTT}_{\it sb}(\mathbf{d}[i])$
}

// compute Bob's Diffie-Hellman contribution

$\mathbf{B} \leftarrow [[0 : 0 \leq i < 64] : 0 \leq j < m]$

\For{$i \in \{0, \ldots, m-1\}$}{
	\For{$j \in \{0, \ldots, m-1\}$}{
		$\mathbf{B}[i] \leftarrow \mathbf{B}[i] + \mathbf{G}[j][i] \circ \mathbf{c}[j]$
	}
	$\mathbf{B}[i] \leftarrow \mathbf{B}[i] + \mathbf{d}[i]$
}

// compute shared noisy one-time pad

$K \leftarrow [\mathsf{sample\_short\_field\_element}() : 0 \leq i < 64]$

$\mathsf{NTT}_{\it sb}(K)$

\For{$i \in \{0, \ldots, m-1\}$}{
	$K \leftarrow K + \mathbf{c}[i] \circ \mathbf{A}[i]$
}

// embed and pad message

$M \leftarrow \mathsf{embed\_msg}(m)$

$\mathsf{NTT}_{\it sb}(M)$

$C \leftarrow K + M$

\textbf{return} {$\mathit{ctxt} = (\mathbf{B}, C)$}
}
\end{scriptsize}
\end{algorithm} 

\begin{algorithm}[!t] \label{algo:tb4-dec-optimized}
\begin{scriptsize}
\caption{\emph{$\mathsf{Dec}$} \textit{(optimized)}}
\KwIn{a secret key $\mathit{sk} = (\mathbf{a}, \mathbf{b})$, a ciphertext $\mathit{ctxt} = (\mathbf{B}, C)$ all represented in frequency domain}
\KwOut{a message $m \in \{0, 1\}^{256}$}
\Begin{
\everypar={\nl}

// compute shared noisy one-time pad

$K \leftarrow [0 : 0 \leq i < 64]$

\For{$i \in \{0, \ldots, m-1\}$}{
	$K \leftarrow K + \mathbf{B}[i] \circ \mathbf{a}[i]$
}

// unpad, intt, and extract message

$M \leftarrow C - K$

$\mathsf{INTT}_{\sf sb}(M)$

\textbf{return} {$\mathsf{extract\_msg}(M)$}
}
\end{scriptsize}
\end{algorithm}

\subsection{Security, Parameters}
\label{section:security-parameters}

The encryption scheme is $\mathsf{IND-CPA}$-secure under the  assumed hardness of a specific member of the decisional Ideal LWE class of problems as described by Bootland \emph{et al.}~\cite{BootlandCSV21}. This specific member is defined below.

\vspace{0.25cm}

\textbf{Hard Problem.} Let $R_p$ and $\Upsilon$ be defined as in \S~\ref{section:native-parameter-sets}. Let $k, l, m$ be small integers. The IMLWE$^*_{k,l,m}$ is to distinguish the distribution (1) from the distribution (2) given the sample $(\mathbf{A}, \mathbf{B}) \in R_p^{k \times l} \times R_p^{k \times m}$ where
\begin{itemize}
\item[(1)] $\mathbf{A}$ is uniformly random and $\mathbf{B} = \mathbf{A} \mathbf{c} + \mathbf{d}$ for some matrices $\mathbf{c} \sim \Upsilon^{l \times m}$ and $\mathbf{d} \sim \Upsilon^{k \times m}$;
\item[(2)] $\mathbf{A}$ and $\mathbf{B}$ are uniformly random.
\end{itemize}

\vspace{0.25cm}

The security reduction constructs a IMLWE solver from a $\mathsf{IND-CPA}$ adversary $\mathcal{A}$. The reduction proceeds in two steps.

In the first step, $\mathcal{A}$ is used to solve a Diffie-Hellman-like problem, where the task is to distinguish the distribution (1) $(\mathbf{G}, \mathbf{A}, \mathbf{B}, K)$ from (2) $(\mathbf{G}, \mathbf{A}, \mathbf{B}, U)$, where all symbols match with their use in the specification of the encryption scheme and where $U \sim \mathcal{U}(R_p)$. In fact, this step is trivial. Supply the adversary with the public parameter $\mathbf{G}$, the public key $\mathbf{A}$, and the ciphertext $(\mathbf{B}, C)$ were $C$ was constructed according to the last lines of $\mathsf{Enc}$. If $\mathcal{A}$ correctly identifies which message was encrypted, then the fourth element of the tuple is not uniform. Conversely, if $\mathcal{A}$'s can do no better than guess at random then the fourth element must be uniform.

In the second step, a solver for this Diffie-Hellman-like problem is used to build one for IMLWE$^*_{m, 1, 1}$. This step is analogous to \S~5.5 of Frodo~\cite{frodo}.

To estimate the hardness of IMLWE$^*_{m, 1, 1}$ we use the tools of Albrecht~\emph{et al.}~\cite{albrecht-estimator} and Ducas~\emph{et al.}~\cite{ducas-estimator}. While these tools estimate the hardness of standard LWE instances of dimension $N$, we argue that they apply also to IMLWE$^*_{m,1,1}$ with $N = 4 m n$, where the factor 4 arises from the integer packing scheme pressing 4 integers into every field element, and the factor $n$ arises from the ring $R_p$.

The aptitude of these estimators warrants a note of caution. The estimators work for a generic $q$-ary lattice. However, the lattice induced by IMLWE$^*_{m,1,1}$ is not $q$-ary and has abundant structure corresponding the algebra over which multiplication is defined. Nevertheless, inspection of the generating matrix shows that the lattice in question is \emph{very close} to $q$-ary, with $q = 2^{16}$. Moreover, the same estimators are used to estimate the hardness of the \emph{structured} (thus not generic) lattice problems underlying Ring- and Module-based lattice cryptosystems. Neither of these caveats are known to give rise to exploitable attacks or even different attack complexity.

\begin{table}
\centering
\caption{Parameters, security, failure probability}
\label{table:parameters}
\begin{tabular}{c||c|c|c|c}
sec. lvl. & $m$ & Albrecht~\emph{et al.} & Ducas~\emph{et al.} & failure probability \\ \hline
128 & 3 & 148.9 & 135.3 & $< 2^{-995\phantom{\vert}}$ \\
192 & 4 & 211.7 & 192.2 & $\sim 2^{-725}$ \\
256 & 6 & 263.5 & 310.8 & $\sim 2^{-331}$
\end{tabular}
\end{table}

\subsection{Homomorphisms and Failure Probability}

The encryption scheme admits two homomorphic operations. First, addition of ciphertexts corresponds to addition of plaintexts modulo 2. Second, multiplication by \emph{sufficiently short} elements of $R_p$ affects the underlying plaintexts in the same way. Both operations can lead to a decryption failure, even if the operand ciphertexts do not, although this event happens with small probability.

To compute an approximation of the probability of decryption failure after any number of homomorphic operations, it is necessary to represent various distributions on a single packed integer. To make this calculation feasible, it is advisable to restrict distributions to the range $[-2^{14},2^{14}]$ by truncation. The probability of decryption failure corresponds to the distance between 1 and the sum of all probabilities of integers in this range. Starting from the distribution of small integers $\Xi$, this distribution evolves as follows.
\begin{itemize}
\item Multiplication of integers corresponds to convolution of their probability distributions.
\item Multiplication of field elements gives rise to at least one packed integer consisting of the sum of 8 products of original packed integers.
\item Multiplication of polynomials in $R_p$ generates a polynomial where each coefficient consists of the sum of two products of field elements.
\item Multiplication of an $l \times m$ matrix of polynomials by an $m \times 1$ vector of polynomials generates an $l \times 1$ vector where each coordinate consists of the sum of $m$ products of polynomials.
\end{itemize}
It is feasible to apply the same homomorphic circuit to the distribution of small elements. The failure probability of decryption of the output ciphertexts is approximately the distance of the sum of this distribution from 1.

