%!TeX root=miden_lattices.tex
\section{Introduction}

The prime number $p = 18446744069414584321 = 2^{64} - 2^{32} + 1 = \Phi_6(2^{32})$ is incredibly useful for proof systems for general-purpose computational integrity. This usefulness originates from several factors:
\begin{itemize}
 \item integers modulo $p$ fit into one 64-bit register;
 \item the field $\mathbb{Z}_p, +, \times$ has a multiplicative subgroup of order $2^{32}$, giving rise to fast arithmetic using NTTs as well as the FRI~\cite{fri} polynomial commitment scheme for constructing transparent SNARKs;
 \item any integer modulo $p$ can be efficiently decomposed into the unique pair of 32-bit limbs;
 \item the product of any two integers less than $2^{32}$ is less than $p$;
 \item the balanced ternary expansion is sparse, and thus gives rise to efficient modular reduction.
\end{itemize}
In light of these compelling properties, $p$ was selected by a number of independent teams for their general purpose computational integrity proof systems, including Polygon Zero, Risc Zero, Triton, and Polygon Miden.

It is often necessary to prove the correct execution of cryptographic algorithms, such as public key encryption or signature verification. The different cost model of proof systems versus physical computers motivates a re-evaluation of the appropriate cryptographic algorithms. In this line of research, Pornin recently proposed ECGFp5~\cite{ecgfp5}, an elliptic curve defined over degree-five extension field of $\mathbb{Z}_p$.

This report focuses on lattice-based cryptography. Specifically:
\begin{itemize}
\item Section~\ref{section:arithmetic} addresses the question how to do the arithmetic for lattice-based cryptography efficiently on virtual machines defined over $\mathbb{Z}_p$, including NTT, memory addressing, and embedding multiple integers into each field element.
\item Section~\ref{section:signatures} discusses supporting the two lattice-based signature schemes selected by the NIST PQC project, Falcon~\cite{falcon} and Dilithium~\cite{dilithium}.
\item Section~\ref{section:encryption} proposes a new lattice-based public key encryption scheme defined over $\mathbb{Z}_p$, which makes use of the embedding technique described above.
\item Section~\ref{section:commitments} proposes a \emph{publicly re-randomizable} commitment scheme based on essentially the same construction.
\end{itemize}
Supporting python code for these techniques and proposed algorithms is available at \url{https://github.com/KULeuven-COSIC/Miden_Lattices}.